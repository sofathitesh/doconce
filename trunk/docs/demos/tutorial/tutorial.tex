%%
%% Automatically generated LaTeX file from Doconce source 
%% http://code.google.com/p/doconce/
%%
\documentclass{article}
\usepackage{hyperref,relsize,epsfig,makeidx,amsmath}
\usepackage[latin1]{inputenc}
\usepackage{ptex2tex}

% Set helvetica as the default font family:
\RequirePackage{helvet}
\renewcommand\familydefault{phv}

\newcommand{\inlinecomment}[2]{  ({\bf #1}: \emph{#2})  }
%\newcommand{\inlinecomment}[2]{}  % turn off inline comments

\makeindex

\begin{document}




\begin{center}
{\LARGE\bf Doconce: Document Once, Include Anywhere}
\end{center}



\begin{center}
{\bf Hans Petter Langtangen${}^{1, 2}$} \\ [0mm]
\end{center}

\begin{center}
{\small ${}^1$Simula Research Laboratory} \\ [-1.0mm]
\end{center}

\begin{center}
{\small ${}^2$University of Oslo} \\ [-1.0mm]
\end{center}

%\vspace{4mm}




\begin{center}
Feb 20, 2011
\end{center}


\begin{itemize}
 \item When writing a note, report, manual, etc., do you find it difficult
   to choose the typesetting format? That is, to choose between plain
   (email-like) text, Wiki, Word/OpenOffice, {\LaTeX}, HTML,
   reStructuredText, Sphinx, XML, etc.  Would it be convenient to
   start with some very simple text-like format that easily converts
   to the formats listed above, and then at some later stage eventually go
   with a particular format?

 \item Do you find it problematic that you have the same information
   scattered around in different documents in different typesetting
   formats? Would it be a good idea to write things once, in one
   place, and include it anywhere?
\end{itemize}

\noindent
If any of these questions are of interest, you should keep on reading.


\section{The Doconce Concept}

Doconce is two things:

\begin{enumerate}
 \item Doconce is a very simple and minimally tagged markup language that
    look like ordinary ASCII text (much like what you would use in an
    email), but the text can be transformed to numerous other formats,
    including HTML, Wiki, {\LaTeX}, PDF, reStructuredText (reST), Sphinx,
    Epytext, and also plain text (where non-obvious formatting/tags are
    removed for clear reading in, e.g., emails). From reStructuredText
    you can go to XML, HTML, {\LaTeX}, PDF, OpenOffice, and from the
    latter to RTF and MS Word.

 \item Doconce is a working strategy for never duplicating information.
    Text is written in a single place and then transformed to
    a number of different destinations of diverse type (software
    source code, manuals, tutorials, books, wikis, memos, emails, etc.).
    The Doconce markup language support this working strategy.
    The slogan is: "Write once, include anywhere".
\end{enumerate}

\noindent
A wide range of markup languages exist. For example, reStructuredText and Sphinx
have recently become popular. So why another one?

\begin{itemize}
  \item Doconce can convert to plain \emph{untagged} text, 
    more desirable for computer programs and email.

  \item Doconce has less cluttered tagging of text.

  \item Doconce has better support for copying in parts of computer code,
    say in examples, directly from the source code files.

  \item Doconce has stronger support for mathematical typesetting, and
    has many features for being integrated with (big) {\LaTeX} projects.

  \item Doconce is almost self-explanatory and is a handy starting point
    for generating documents in more complicated markup languages, such
    as Google Wiki, {\LaTeX}, and Sphinx. A primary application of Doconce
    is just to make the initial versions of a Sphinx or Wiki document.
\end{itemize}

\noindent
Doconce was particularly written for the following sample applications:

\begin{itemize}
  \item Large books written in {\LaTeX}, but where many pieces (computer demos,
    projects, examples) can be written in Doconce to appear in other
    contexts in other formats, including plain HTML, Sphinx, or MS Word.

  \item Software documentation, primarily Python doc strings, which one wants
    to appear as plain untagged text for viewing in Pydoc, as reStructuredText
    for use with Sphinx, as wiki text when publishing the software at
    googlecode.com, and as {\LaTeX} integrated in, e.g., a thesis.

  \item Quick memos, which start as plain text in email, then some small
    amount of Doconce tagging is added, before the memos can appear as
    MS Word documents or in wikis.
\end{itemize}

\noindent
Disclaimer: Doconce is a simple tool, largely based on interpreting
and handling text through regular expressions. The possibility for
tweaking the layout is obviously limited since the text can go to
all sorts of sophisticated markup languages. Moreover, because of
limitations of regular expressions, some formatting may face problems 
when transformed to other formats.


\section{What Does Doconce Look Like?}

Doconce text looks like ordinary text, but there are some almost invisible
text constructions that allow you to control the formating. For example,

\begin{itemize}
  \item bullet lists arise from lines starting with an asterisk,

  \item \emph{emphasized words} are surrounded by asterisks, 

  \item \textbf{words in boldface} are surrounded by underscores, 

  \item words from computer code are enclosed in back quotes and 
    then typeset verbatim,

  \item blocks of computer code can easily be included, also from source files,

  \item blocks of {\LaTeX} mathematics can easily be included,

  \item there is support for both {\LaTeX} and text-like inline mathematics,

  \item figures with captions, URLs with links, labels and references
    are supported,

  \item comments can be inserted throughout the text,

  \item with a simple preprocessor, which is integrated, one can include
    other documents (files) and large portions of text can be defined
    in or out of the text,

  \item with the Mako preprocessor one can even embed Python
    code and use this to steer generation of Doconce text.
\end{itemize}

\noindent
Here is an example of some simple text written in the Doconce format:
\begin{Verbatim}[fontsize=\fontsize{9pt}{9pt},tabsize=8,baselinestretch=0.85,
fontfamily=tt,xleftmargin=7mm]
===== A Subsection with Sample Text =====
label{my:first:sec}

Ordinary text looks like ordinary text, and the tags used for
_boldface_ words, *emphasized* words, and `computer` words look
natural in plain text.  Lists are typeset as you would do in an email,

  * item 1
  * item 2
  * item 3

Lists can also have automatically numbered items instead of bullets,

  o item 1
  o item 2
  o item 3

URLs with a link word are possible, as in "hpl":"http://folk.uio.no/hpl".
If the word is URL, the URL itself becomes the link name,
as in "URL":"tutorial.do.txt".

References to sections may use logical names as labels (e.g., a
"label" command right after the section title), as in the reference to
Chapter ref{my:first:sec}. 

Doconce also allows inline comments such as [hpl: here I will make
some remarks to the text] for allowing authors to make notes. Inline
comments can be removed from the output by a command-line argument
(see Chapter ref{doconce2formats} for an example).

Tables are also supperted, e.g.,

  |--------------------------------|
  |time  | velocity | acceleration |
  |--------------------------------|
  | 0.0  | 1.4186   | -5.01        |
  | 2.0  | 1.376512 | 11.919       |
  | 4.0  | 1.1E+1   | 14.717624    |
  |--------------------------------|

# lines beginning with # are comment lines
\end{Verbatim}
\noindent
The Doconce text above results in the following little document:

\subsection{A Subsection with Sample Text}

\label{my:first:sec}

Ordinary text looks like ordinary text, and the tags used for
\textbf{boldface} words, \emph{emphasized} words, and {\fontsize{10pt}{10pt}\verb!computer!} words look
natural in plain text.  Lists are typeset as you would do in an email,

\begin{itemize}
  \item item 1

  \item item 2

  \item item 3
\end{itemize}

\noindent
Lists can also have numbered items instead of bullets, just use an {\fontsize{10pt}{10pt}\verb!o!}
(for ordered) instead of the asterisk:

\begin{enumerate}
 \item item 1

 \item item 2

 \item item 3
\end{enumerate}

\noindent
URLs with a link word are possible, as in \href{http://folk.uio.no/hpl}{hpl}.
If the word is URL, the URL itself becomes the link name,
as in \href{tutorial.do.txt}{tutorial.do.txt}.

References to sections may use logical names as labels (e.g., a
"label" command right after the section title), as in the reference to
Chapter~\ref{my:first:sec}. 

Doconce also allows inline comments such as \inlinecomment{hpl}{here I will make
some remarks to the text} for allowing authors to make notes. Inline
comments can be removed from the output by a command-line argument
(see Chapter~\ref{doconce2formats} for an example).

Tables are also supperted, e.g.,


\begin{quote}\begin{tabular}{ccc}
\hline
\multicolumn{1}{c}{time} & \multicolumn{1}{c}{velocity} & \multicolumn{1}{c}{acceleration} \\
\hline
0.0          & 1.4186       & -5.01        \\
2.0          & 1.376512     & 11.919       \\
4.0          & 1.1E+1       & 14.717624    \\
\hline
\end{tabular}\end{quote}

\noindent

\subsection{Mathematics and Computer Code}

Inline mathematics, such as $\nu = \sin(x)$,
allows the formula to be specified both as {\LaTeX} and as plain text.
This results in a professional {\LaTeX} typesetting, but in other formats
the text version normally looks better than raw {\LaTeX} mathematics with
backslashes. An inline formula like $\nu = \sin(x)$ is
typeset as
\begin{Verbatim}[fontsize=\fontsize{9pt}{9pt},tabsize=8,baselinestretch=0.85,
fontfamily=tt,xleftmargin=7mm]
$\nu = \sin(x)$|$v = sin(x)$
\end{Verbatim}
\noindent
The pipe symbol acts as a delimiter between {\LaTeX} code and the plain text
version of the formula.

Blocks of mathematics are better typeset with raw {\LaTeX}, inside
{\fontsize{10pt}{10pt}\verb!!bt!} and {\fontsize{10pt}{10pt}\verb!!et!} (begin tex / end tex) instructions. 
The result looks like this:
\begin{eqnarray}
{\partial u\over\partial t} &=& \nabla^2 u + f, \label{myeq1}\\
{\partial v\over\partial t} &=& \nabla\cdot(q(u)\nabla v) + g
\end{eqnarray}
Of course, such blocks only looks nice in {\LaTeX}. The raw
{\LaTeX} syntax appears in all other formats (but can still be useful
for those who can read {\LaTeX} syntax).

You can have blocks of computer code, starting and ending with
{\fontsize{10pt}{10pt}\verb!!bc!} and {\fontsize{10pt}{10pt}\verb!!ec!} instructions, respectively. Such blocks look like
\providecommand{\shadedskip}{}
\definecolor{shadecolor}{rgb}{0.87843, 0.95686, 1.0}
\renewenvironment{shadedskip}{
\def\FrameCommand{\colorbox{shadecolor}}\FrameRule0.6pt
\MakeFramed {\FrameRestore}\vskip3mm}{\vskip0mm\endMakeFramed}
\providecommand{\shadedquoteBlue}{}
\renewenvironment{shadedquoteBlue}[1][]{
\bgroup\rmfamily
\fboxsep=0mm\relax
\begin{shadedskip}
\list{}{\parsep=-2mm\parskip=0mm\topsep=0pt\leftmargin=2mm
\rightmargin=2\leftmargin\leftmargin=4pt\relax}
\item\relax}
{\endlist\end{shadedskip}\egroup}\begin{shadedquoteBlue}
\fontsize{9pt}{9pt}
\begin{Verbatim}
from math import sin, pi
def myfunc(x):
    return sin(pi*x)

import integrate
I = integrate.trapezoidal(myfunc, 0, pi, 100)
\end{Verbatim}
\end{shadedquoteBlue}
\noindent
It is possible to add a specification of a (ptex2tex-style)
environment for typesetting the verbatim code block, e.g., {\fontsize{10pt}{10pt}\verb!!bc xxx!}
where {\fontsize{10pt}{10pt}\verb!xxx!} is an identifier like {\fontsize{10pt}{10pt}\verb!pycod!} for code snippet in Python,
{\fontsize{10pt}{10pt}\verb!sys!} for terminal session, etc. When Doconce is filtered to {\LaTeX},
these identifiers are used as in ptex2tex and defined in a
configuration file {\fontsize{10pt}{10pt}\verb!.ptext2tex.cfg!}, while when filtering
to Sphinx, one can have a comment line in the Doconce file for
mapping the identifiers to legal language names for Sphinx (which equals
the legal language names for Pygments):
\begin{Verbatim}[fontsize=\fontsize{9pt}{9pt},tabsize=8,baselinestretch=0.85,
fontfamily=tt,xleftmargin=7mm]
# sphinx code-blocks: pycod=python cod=py cppcod=c++ sys=console
\end{Verbatim}
\noindent
By default, {\fontsize{10pt}{10pt}\verb!pro!} and {\fontsize{10pt}{10pt}\verb!cod!} are {\fontsize{10pt}{10pt}\verb!python!}, {\fontsize{10pt}{10pt}\verb!sys!} is {\fontsize{10pt}{10pt}\verb!console!},
while {\fontsize{10pt}{10pt}\verb!xpro!} and {\fontsize{10pt}{10pt}\verb!xcod!} are computer language specific for {\fontsize{10pt}{10pt}\verb!x!}
in {\fontsize{10pt}{10pt}\verb!f!} (Fortran), {\fontsize{10pt}{10pt}\verb!c!} (C), {\fontsize{10pt}{10pt}\verb!cpp!} (C++), and {\fontsize{10pt}{10pt}\verb!py!} (Python).
% {\fontsize{10pt}{10pt}\verb!rb!} (Ruby), {\fontsize{10pt}{10pt}\verb!pl!} (Perl), and {\fontsize{10pt}{10pt}\verb!sh!} (Unix shell).

% (Any sphinx code-block comment, whether inside verbatim code
% blocks or outside, yields a mapping between bc arguments
% and computer languages. In case of muliple definitions, the
% first one is used.)

One can also copy computer code directly from files, either the
complete file or specified parts.  Computer code is then never
duplicated in the documentation (important for the principle of
avoiding copying information!). A complete file is typeset 
with {\fontsize{10pt}{10pt}\verb!!bc pro!}, while a part of a file is copied into a {\fontsize{10pt}{10pt}\verb!!bc cod!}
environment. What {\fontsize{10pt}{10pt}\verb!pro!} and {\fontsize{10pt}{10pt}\verb!cod!} mean is then defined through
a {\fontsize{10pt}{10pt}\verb!.ptex2tex.cfg!} file for {\LaTeX} and a {\fontsize{10pt}{10pt}\verb!sphinx code-blocks!}
comment for Sphinx.

Another document can be included by writing {\fontsize{10pt}{10pt}\verb!#include "mynote.do.txt"!}
on a line starting with (another) hash sign.  Doconce documents have
extension {\fontsize{10pt}{10pt}\verb!do.txt!}. The {\fontsize{10pt}{10pt}\verb!do!} part stands for doconce, while the
trailing {\fontsize{10pt}{10pt}\verb!.txt!} denotes a text document so that editors gives you the
right writing enviroment for plain text.

\subsection{Macros (Newcommands), Cross-References, Index, and Bibliography}

\label{newcommands}

Doconce supports a type of macros via a {\LaTeX}-style \emph{newcommand}
construction.  The newcommands defined in a file with name
{\fontsize{10pt}{10pt}\verb!newcommand_replace.tex!} are expanded when Doconce is filtered to
other formats, except for {\LaTeX} (since {\LaTeX} performs the expansion
itself).  Newcommands in files with names {\fontsize{10pt}{10pt}\verb!newcommands.tex!} and
{\fontsize{10pt}{10pt}\verb!newcommands_keep.tex!} are kept unaltered when Doconce text is
filtered to other formats, except for the Sphinx format. Since Sphinx
understands {\LaTeX} math, but not newcommands if the Sphinx output is
HTML, it makes most sense to expand all newcommands.  Normally, a user
will put all newcommands that appear in math blocks surrounded by
{\fontsize{10pt}{10pt}\verb!!bt!} and {\fontsize{10pt}{10pt}\verb!!et!} in {\fontsize{10pt}{10pt}\verb!newcommands_keep.tex!} to keep them unchanged, at
least if they contribute to make the raw {\LaTeX} math text easier to
read in the formats that cannot render {\LaTeX}.  Newcommands used
elsewhere throughout the text will usually be placed in
{\fontsize{10pt}{10pt}\verb!newcommands_replace.tex!} and expanded by Doconce.  The definitions of
newcommands in the {\fontsize{10pt}{10pt}\verb!newcommands*.tex!} files \emph{must} appear on a single
line (multi-line newcommands are too hard to parse with regular
expressions).

Recent versions of Doconce also offer cross referencing, typically one
can define labels below (sub)sections, in figure captions, or in
equations, and then refer to these later. Entries in an index can be
defined and result in an index at the end for the {\LaTeX} and Sphinx
formats. Citations to literature, with an accompanying bibliography in
a file, are also supported. The syntax of labels, references,
citations, and the bibliography closely resembles that of {\LaTeX},
making it easy for Doconce documents to be integrated in {\LaTeX}
projects (manuals, books). For further details on functionality and
syntax we refer to the {\fontsize{10pt}{10pt}\verb!docs/manual/manual.do.txt!} file (see the
\href{https://doconce.googlecode.com/hg/trunk/docs/demos/manual/index.html}{demo
page} for various formats of this document).


% Example on including another Doconce file (using preprocess):


\section{From Doconce to Other Formats}

\label{doconce2formats}

Transformation of a Doconce document to various other
formats applies the script {\fontsize{10pt}{10pt}\verb!doconce format!}:
\vspace{4pt}
\begin{Verbatim}[numbers=none,frame=lines,label=\fbox{{\tiny Terminal}},fontsize=\fontsize{9pt}{9pt},
labelposition=topline,framesep=2.5mm,framerule=0.7pt]
Unix/DOS> doconce format format mydoc.do.txt
\end{Verbatim}
The {\fontsize{10pt}{10pt}\verb!preprocess!} program is always used to preprocess the file first,
and options to {\fontsize{10pt}{10pt}\verb!preprocess!} can be added after the filename. For example,
\vspace{4pt}
\begin{Verbatim}[numbers=none,frame=lines,label=\fbox{{\tiny Terminal}},fontsize=\fontsize{9pt}{9pt},
labelposition=topline,framesep=2.5mm,framerule=0.7pt]
Unix/DOS> doconce format LaTeX mydoc.do.txt -Dextra_sections
\end{Verbatim}
The variable {\fontsize{10pt}{10pt}\verb!FORMAT!} is always defined as the current format when
running {\fontsize{10pt}{10pt}\verb!preprocess!}. That is, in the last example, {\fontsize{10pt}{10pt}\verb!FORMAT!} is
defined as {\fontsize{10pt}{10pt}\verb!LaTeX!}. Inside the Doconce document one can then perform
format specific actions through tests like {\fontsize{10pt}{10pt}\verb!#if FORMAT == "LaTeX"!}.

Inline comments in the text are removed from the output by
\vspace{4pt}
\begin{Verbatim}[numbers=none,frame=lines,label=\fbox{{\tiny Terminal}},fontsize=\fontsize{9pt}{9pt},
labelposition=topline,framesep=2.5mm,framerule=0.7pt]
Unix/DOS> doconce format LaTeX mydoc.do.txt remove_inline_comments
\end{Verbatim}
One can also remove such comments from the original Doconce file
by running a helper script in the {\fontsize{10pt}{10pt}\verb!bin!} folder of the Doconce
source code:
\begin{Verbatim}[fontsize=\fontsize{9pt}{9pt},tabsize=8,baselinestretch=0.85,
fontfamily=tt,xleftmargin=7mm]
Unix/DOS> doconce remove_inline_comments mydoc.do.txt
\end{Verbatim}
\noindent
This action is convenient when a Doconce document reaches its final form.

\subsection{HTML}

Making an HTML version of a Doconce file {\fontsize{10pt}{10pt}\verb!mydoc.do.txt!}
is performed by
\vspace{4pt}
\begin{Verbatim}[numbers=none,frame=lines,label=\fbox{{\tiny Terminal}},fontsize=\fontsize{9pt}{9pt},
labelposition=topline,framesep=2.5mm,framerule=0.7pt]
Unix/DOS> doconce format HTML mydoc.do.txt
\end{Verbatim}
The resulting file {\fontsize{10pt}{10pt}\verb!mydoc.html!} can be loaded into any web browser for viewing.

\subsection{{\LaTeX}}

Making a {\LaTeX} file {\fontsize{10pt}{10pt}\verb!mydoc.tex!} from {\fontsize{10pt}{10pt}\verb!mydoc.do.txt!} is done in two steps:
% Note: putting code blocks inside a list is not successful in many
% formats - the text may be messed up. A better choice is a paragraph
% environment, as used here.

\paragraph{Step 1.}
Filter the doconce text to a pre-{\LaTeX} form {\fontsize{10pt}{10pt}\verb!mydoc.p.tex!} for
     {\fontsize{10pt}{10pt}\verb!ptex2tex!}:
\vspace{4pt}
\begin{Verbatim}[numbers=none,frame=lines,label=\fbox{{\tiny Terminal}},fontsize=\fontsize{9pt}{9pt},
labelposition=topline,framesep=2.5mm,framerule=0.7pt]
Unix/DOS> doconce format LaTeX mydoc.do.txt
\end{Verbatim}
{\LaTeX}-specific commands ("newcommands") in math formulas and similar
can be placed in files {\fontsize{10pt}{10pt}\verb!newcommands.tex!}, {\fontsize{10pt}{10pt}\verb!newcommands_keep.tex!}, or
{\fontsize{10pt}{10pt}\verb!newcommands_replace.tex!} (see Section~\ref{newcommands}). 
If these files are present, they are included in the {\LaTeX} document 
so that your commands are defined.

\paragraph{Step 2.}
Run {\fontsize{10pt}{10pt}\verb!ptex2tex!} (if you have it) to make a standard {\LaTeX} file,
\vspace{4pt}
\begin{Verbatim}[numbers=none,frame=lines,label=\fbox{{\tiny Terminal}},fontsize=\fontsize{9pt}{9pt},
labelposition=topline,framesep=2.5mm,framerule=0.7pt]
Unix/DOS> ptex2tex mydoc
\end{Verbatim}
or just perform a plain copy,
\vspace{4pt}
\begin{Verbatim}[numbers=none,frame=lines,label=\fbox{{\tiny Terminal}},fontsize=\fontsize{9pt}{9pt},
labelposition=topline,framesep=2.5mm,framerule=0.7pt]
Unix/DOS> cp mydoc.p.tex mydoc.tex
\end{Verbatim}
Doconce generates a {\fontsize{10pt}{10pt}\verb!.p.tex!} file with some preprocessor macros.
For example, to enable font Helvetica instead of the standard
Computer Modern font,
\vspace{4pt}
\begin{Verbatim}[numbers=none,frame=lines,label=\fbox{{\tiny Terminal}},fontsize=\fontsize{9pt}{9pt},
labelposition=topline,framesep=2.5mm,framerule=0.7pt]
Unix/DOS> ptex2tex -DHELVETICA mydoc
\end{Verbatim}
The title, authors, and date are by default typeset in a non-standard
way to enable a nicer treatment of multiple authors having
institutions in common. The standard {\LaTeX} "maketitle" heading
is also available through
\vspace{4pt}
\begin{Verbatim}[numbers=none,frame=lines,label=\fbox{{\tiny Terminal}},fontsize=\fontsize{9pt}{9pt},
labelposition=topline,framesep=2.5mm,framerule=0.7pt]
Unix/DOS> ptex2tex -DTRAD_LATEX_HEADING mydoc
\end{Verbatim}

The {\fontsize{10pt}{10pt}\verb!ptex2tex!} tool makes it possible to easily switch between many
different fancy formattings of computer or verbatim code in {\LaTeX}
documents. After any {\fontsize{10pt}{10pt}\verb!!bc sys!} command in the Doconce source you can
insert verbatim block styles as defined in your {\fontsize{10pt}{10pt}\verb!.ptex2tex.cfg!}
file, e.g., {\fontsize{10pt}{10pt}\verb!!bc sys cod!} for a code snippet, where {\fontsize{10pt}{10pt}\verb!cod!} is set to
a certain environment in {\fontsize{10pt}{10pt}\verb!.ptex2tex.cfg!} (e.g., {\fontsize{10pt}{10pt}\verb!CodeIntended!}).
There are over 30 styles to choose from.

\paragraph{Step 3.}
Compile {\fontsize{10pt}{10pt}\verb!mydoc.tex!}
and create the PDF file:
\vspace{4pt}
\begin{Verbatim}[numbers=none,frame=lines,label=\fbox{{\tiny Terminal}},fontsize=\fontsize{9pt}{9pt},
labelposition=topline,framesep=2.5mm,framerule=0.7pt]
Unix/DOS> latex mydoc
Unix/DOS> latex mydoc
Unix/DOS> makeindex mydoc   # if index
Unix/DOS> bibitem mydoc     # if bibliography
Unix/DOS> latex mydoc
Unix/DOS> dvipdf mydoc
\end{Verbatim}
If one wishes to use the {\fontsize{10pt}{10pt}\verb!Minted_Python!}, {\fontsize{10pt}{10pt}\verb!Minted_Cpp!}, etc., environments
in {\fontsize{10pt}{10pt}\verb!ptex2tex!} for typesetting code, the {\fontsize{10pt}{10pt}\verb!minted!} {\LaTeX} package is needed.
This package is included by running {\fontsize{10pt}{10pt}\verb!doconce format!} with the
{\fontsize{10pt}{10pt}\verb!-DMINTED!} option:
\vspace{4pt}
\begin{Verbatim}[numbers=none,frame=lines,label=\fbox{{\tiny Terminal}},fontsize=\fontsize{9pt}{9pt},
labelposition=topline,framesep=2.5mm,framerule=0.7pt]
Unix/DOS> ptex2tex -DMINTED mydoc
\end{Verbatim}
In this case, {\fontsize{10pt}{10pt}\verb!latex!} must be run with the
{\fontsize{10pt}{10pt}\verb!-shell-escape!} option:
\vspace{4pt}
\begin{Verbatim}[numbers=none,frame=lines,label=\fbox{{\tiny Terminal}},fontsize=\fontsize{9pt}{9pt},
labelposition=topline,framesep=2.5mm,framerule=0.7pt]
Unix/DOS> latex -shell-escape mydoc
Unix/DOS> latex -shell-escape mydoc
Unix/DOS> makeindex mydoc   # if index
Unix/DOS> bibitem mydoc     # if bibliography
Unix/DOS> latex -shell-escape mydoc
Unix/DOS> dvipdf mydoc
\end{Verbatim}
The {\fontsize{10pt}{10pt}\verb!-shell-escape!} option is required because the {\fontsize{10pt}{10pt}\verb!minted.sty!} style
file runs the {\fontsize{10pt}{10pt}\verb!pygments!} program to format code, and this program
cannot be run from {\fontsize{10pt}{10pt}\verb!latex!} without the {\fontsize{10pt}{10pt}\verb!-shell-escape!} option.

\subsection{Plain ASCII Text}

We can go from Doconce "back to" plain untagged text suitable for viewing
in terminal windows, inclusion in email text, or for insertion in
computer source code:
\vspace{4pt}
\begin{Verbatim}[numbers=none,frame=lines,label=\fbox{{\tiny Terminal}},fontsize=\fontsize{9pt}{9pt},
labelposition=topline,framesep=2.5mm,framerule=0.7pt]
Unix/DOS> doconce format plain mydoc.do.txt  # results in mydoc.txt
\end{Verbatim}

\subsection{reStructuredText}

Going from Doconce to reStructuredText gives a lot of possibilities to
go to other formats. First we filter the Doconce text to a
reStructuredText file {\fontsize{10pt}{10pt}\verb!mydoc.rst!}:
\vspace{4pt}
\begin{Verbatim}[numbers=none,frame=lines,label=\fbox{{\tiny Terminal}},fontsize=\fontsize{9pt}{9pt},
labelposition=topline,framesep=2.5mm,framerule=0.7pt]
Unix/DOS> doconce format rst mydoc.do.txt
\end{Verbatim}
We may now produce various other formats:
\vspace{4pt}
\begin{Verbatim}[numbers=none,frame=lines,label=\fbox{{\tiny Terminal}},fontsize=\fontsize{9pt}{9pt},
labelposition=topline,framesep=2.5mm,framerule=0.7pt]
Unix/DOS> rst2html.py  mydoc.rst > mydoc.html # HTML
Unix/DOS> rst2latex.py mydoc.rst > mydoc.tex  # LaTeX
Unix/DOS> rst2xml.py   mydoc.rst > mydoc.xml  # XML
Unix/DOS> rst2odt.py   mydoc.rst > mydoc.odt  # OpenOffice
\end{Verbatim}
The OpenOffice file {\fontsize{10pt}{10pt}\verb!mydoc.odt!} can be loaded into OpenOffice and
saved in, among other things, the RTF format or the Microsoft Word format.
That is, one can easily go from Doconce to Microsoft Word.

\subsection{Sphinx}

Sphinx documents can be created from a Doconce source in a few steps.

\paragraph{Step 1.}
Translate Doconce into the Sphinx dialect of
the reStructuredText format:
\vspace{4pt}
\begin{Verbatim}[numbers=none,frame=lines,label=\fbox{{\tiny Terminal}},fontsize=\fontsize{9pt}{9pt},
labelposition=topline,framesep=2.5mm,framerule=0.7pt]
Unix/DOS> doconce format sphinx mydoc.do.txt
\end{Verbatim}

\paragraph{Step 2.}
Create a Sphinx root directory with a {\fontsize{10pt}{10pt}\verb!conf.py!} file, 
either manually or by using the interactive {\fontsize{10pt}{10pt}\verb!sphinx-quickstart!}
program. Here is a scripted version of the steps with the latter:
\vspace{4pt}
\begin{Verbatim}[numbers=none,frame=lines,label=\fbox{{\tiny Terminal}},fontsize=\fontsize{9pt}{9pt},
labelposition=topline,framesep=2.5mm,framerule=0.7pt]
mkdir sphinx-rootdir
sphinx-quickstart <<EOF
sphinx-rootdir
n
_
Name of My Sphinx Document
Author
version
version
.rst
index
n
y
n
n
n
n
y
n
n
y
y
y
EOF
\end{Verbatim}
These statements are automated by the command
\vspace{4pt}
\begin{Verbatim}[numbers=none,frame=lines,label=\fbox{{\tiny Terminal}},fontsize=\fontsize{9pt}{9pt},
labelposition=topline,framesep=2.5mm,framerule=0.7pt]
Unix/DOS> doconce sphinx_dir mydoc.do.txt
\end{Verbatim}

\paragraph{Step 3.}
Move the {\fontsize{10pt}{10pt}\verb!tutorial.rst!} file to the Sphinx root directory:
\vspace{4pt}
\begin{Verbatim}[numbers=none,frame=lines,label=\fbox{{\tiny Terminal}},fontsize=\fontsize{9pt}{9pt},
labelposition=topline,framesep=2.5mm,framerule=0.7pt]
Unix/DOS> mv mydoc.rst sphinx-rootdir
\end{Verbatim}
If you have figures in your document, the relative paths to those will
be invalid when you work with {\fontsize{10pt}{10pt}\verb!mydoc.rst!} in the {\fontsize{10pt}{10pt}\verb!sphinx-rootdir!}
directory. Either edit {\fontsize{10pt}{10pt}\verb!mydoc.rst!} so that figure file paths are correct,
or simply copy your figure directory to {\fontsize{10pt}{10pt}\verb!sphinx-rootdir!} (if all figures
are located in a subdirectory).

\paragraph{Step 4.}
Edit the generated {\fontsize{10pt}{10pt}\verb!index.rst!} file so that {\fontsize{10pt}{10pt}\verb!mydoc.rst!}
is included, i.e., add {\fontsize{10pt}{10pt}\verb!mydoc!} to the {\fontsize{10pt}{10pt}\verb!toctree!} section so that it becomes
\begin{Verbatim}[fontsize=\fontsize{9pt}{9pt},tabsize=8,baselinestretch=0.85,
fontfamily=tt,xleftmargin=7mm]
.. toctree::
   :maxdepth: 2

   mydoc
\end{Verbatim}
\noindent
(The spaces before {\fontsize{10pt}{10pt}\verb!mydoc!} are important!)

\paragraph{Step 5.}
Generate, for instance, an HTML version of the Sphinx source:
\vspace{4pt}
\begin{Verbatim}[numbers=none,frame=lines,label=\fbox{{\tiny Terminal}},fontsize=\fontsize{9pt}{9pt},
labelposition=topline,framesep=2.5mm,framerule=0.7pt]
make clean   # remove old versions
make html
\end{Verbatim}
Many other formats are also possible.

\paragraph{Step 6.}
View the result:
\vspace{4pt}
\begin{Verbatim}[numbers=none,frame=lines,label=\fbox{{\tiny Terminal}},fontsize=\fontsize{9pt}{9pt},
labelposition=topline,framesep=2.5mm,framerule=0.7pt]
Unix/DOS> firefox _build/html/index.html
\end{Verbatim}

Note that verbatim code blocks can be typeset in a variety of ways
depending the argument that follows {\fontsize{10pt}{10pt}\verb!!bc!}: {\fontsize{10pt}{10pt}\verb!cod!} gives Python
({\fontsize{10pt}{10pt}\verb!code-block:: python!} in Sphinx syntax) and {\fontsize{10pt}{10pt}\verb!cppcod!} gives C++, but
all such arguments can be customized both for Sphinx and {\LaTeX} output.

% Desired extension: sphinx can utilize a "pycod" or "c++cod"
% instruction as currently done in latex for ptex2tex and write
% out the right code block name accordingly.

\subsection{Google Code Wiki}

There are several different wiki dialects, but Doconce only support the
one used by \href{http://code.google.com/p/support/wiki/WikiSyntax}{Google Code}.
The transformation to this format, called {\fontsize{10pt}{10pt}\verb!gwiki!} to explicitly mark
it as the Google Code dialect, is done by
\vspace{4pt}
\begin{Verbatim}[numbers=none,frame=lines,label=\fbox{{\tiny Terminal}},fontsize=\fontsize{9pt}{9pt},
labelposition=topline,framesep=2.5mm,framerule=0.7pt]
Unix/DOS> doconce format gwiki mydoc.do.txt
\end{Verbatim}
You can then open a new wiki page for your Google Code project, copy
the {\fontsize{10pt}{10pt}\verb!mydoc.gwiki!} output file from {\fontsize{10pt}{10pt}\verb!doconce format!} and paste the
file contents into the wiki page. Press \textbf{Preview} or \textbf{Save Page} to
see the formatted result.

When the Doconce file contains figures, each figure filename must be
replaced by a URL where the figure is available. There are instructions
in the file for doing this. Usually, one performs this substitution
automatically (see next section).

\subsection{Tweaking the Doconce Output}

Occasionally, one would like to tweak the output in a certain format
from Doconce. One example is figure filenames when transforming
Doconce to reStructuredText. Since Doconce does not know if the
{\fontsize{10pt}{10pt}\verb!.rst!} file is going to be filtered to {\LaTeX} or HTML, it cannot know
if {\fontsize{10pt}{10pt}\verb!.eps!} or {\fontsize{10pt}{10pt}\verb!.png!} is the most appropriate image filename.
The solution is to use a text substitution command or code with, e.g., sed,
perl, python, or scitools subst, to automatically edit the output file
from Doconce. It is then wise to run Doconce and the editing commands
from a script to automate all steps in going from Doconce to the final
format(s). The {\fontsize{10pt}{10pt}\verb!make.sh!} files in {\fontsize{10pt}{10pt}\verb!docs/manual!} and {\fontsize{10pt}{10pt}\verb!docs/tutorial!} 
constitute comprehensive examples on how such scripts can be made.

\subsection{Demos}

The current text is generated from a Doconce format stored in the file
\begin{Verbatim}[fontsize=\fontsize{9pt}{9pt},tabsize=8,baselinestretch=0.85,
fontfamily=tt,xleftmargin=7mm]
docs/tutorial/tutorial.do.txt
\end{Verbatim}
\noindent
The file {\fontsize{10pt}{10pt}\verb!make.sh!} in the {\fontsize{10pt}{10pt}\verb!tutorial!} directory of the
Doconce source code contains a demo of how to produce a variety of
formats.  The source of this tutorial, {\fontsize{10pt}{10pt}\verb!tutorial.do.txt!} is the
starting point.  Running {\fontsize{10pt}{10pt}\verb!make.sh!} and studying the various generated
files and comparing them with the original {\fontsize{10pt}{10pt}\verb!tutorial.do.txt!} file,
gives a quick introduction to how Doconce is used in a real case.
\href{https://doconce.googlecode.com/hg/trunk/docs/demos/tutorial/index.html}{Here} 
is a sample of how this tutorial looks in different formats.

There is another demo in the {\fontsize{10pt}{10pt}\verb!docs/manual!} directory which
translates the more comprehensive documentation, {\fontsize{10pt}{10pt}\verb!manual.do.txt!}, to
various formats. The {\fontsize{10pt}{10pt}\verb!make.sh!} script runs a set of translations.

\subsection{Dependencies}

If you make use of preprocessor directives in the Doconce source,
either \href{http://code.google.com/p/preprocess}{Preprocess} or \href{http://www.makotemplates.org}{Mako} must be installed.  To make {\LaTeX}
documents (without going through the reStructuredText format) you also
need \href{http://code.google.com/p/ptex2tex}{ptex2tex} and some style
files that {\fontsize{10pt}{10pt}\verb!ptex2tex!} potentially makes use of.  Going from
reStructuredText to formats such as XML, OpenOffice, HTML, and {\LaTeX}
requires \href{http://docutils.sourceforge.net}{docutils}.  Making Sphinx
documents requires of course \href{http://sphinx.pocoo.org}{Sphinx}.
All of the mentioned potential dependencies are pure Python packages
which are easily installed.


\section{Warning/Disclaimer}

Doconce can be viewed is a unified interface to a variety of
typesetting formats.  This interface is minimal in the sense that a
lot of typesetting features are not supported, for example, footnotes
and bibliography. For many documents the simple Doconce format is
sufficient, while in other cases you need more sophisticated
formats. Then you can just filter the Doconce text to a more
approprite format and continue working in this format only.  For
example, reStructuredText is a good alternative: it is more tagged
than Doconce and cannot be filtered to plain, untagged text, or wiki,
and the {\LaTeX} output is not at all as clean, but it also has a lot
more typesetting and tagging features than Doconce.

\printindex

\end{document}