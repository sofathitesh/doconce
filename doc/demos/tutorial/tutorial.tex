%%
%% This file is automatically generated from doconce source
%%
%% doconce: http://code.google.com/p/doconce/
%%




%-------------------------- begin preamble --------------------------
\documentclass[%
oneside,                 % oneside: electronic viewing, twoside: printing
final,                   % or draft (marks overfull hboxes)
10pt]{article}

\listfiles               % print all files needed to compile this document



\usepackage{relsize,epsfig,makeidx,amsmath,amsfonts}
\usepackage[latin1]{inputenc}
\usepackage{ptex2tex}
\usepackage{minted}  % requires latex/pdflatex -shell-escape (to run pygments)
\usemintedstyle{default}

% Set helvetica as the default font family:
\RequirePackage{helvet}
\renewcommand\familydefault{phv}

% Hyperlinks in PDF:
\usepackage[%
    colorlinks=true,
    linkcolor=black,
    %linkcolor=blue,
    citecolor=black,
    filecolor=black,
    %filecolor=blue,
    urlcolor=black,
    pdfmenubar=true,
    pdftoolbar=true,
    urlcolor=black,
    %urlcolor=blue,
    bookmarksdepth=3   % Uncomment (and tweak) for PDF bookmarks with more levels than the TOC
            ]{hyperref}
%\hyperbaseurl{}   % hyperlinks are relative to this root

\setcounter{tocdepth}{2}

% Tricks for having figures close to where they are defined:
% 1. define less restrictive rules for where to put figures
\setcounter{topnumber}{2}
\setcounter{bottomnumber}{2}
\setcounter{totalnumber}{4}
\renewcommand{\topfraction}{0.85}
\renewcommand{\bottomfraction}{0.85}
\renewcommand{\textfraction}{0.15}
\renewcommand{\floatpagefraction}{0.7}
% 2. ensure all figures are flushed before next section
\usepackage[section]{placeins}
% 3. enable begin{figure}[H] (often leads to ugly pagebreaks)
%\usepackage{float}\restylefloat{figure}

\newenvironment{exercise}{}{}
\newcounter{exerno}

\newcommand{\inlinecomment}[2]{  ({\bf #1}: \emph{#2})  }
%\newcommand{\inlinecomment}[2]{}  % turn off inline comments

% insert custom LaTeX commands...

\makeindex

\begin{document}
%-------------------------- end preamble --------------------------


% Missing: FIGURE, MOVIE, environments



% ----------------- title -------------------------

\begin{center}
{\LARGE\bf Doconce: Document Once, Include \\ [1.5mm] Anywhere}
\end{center}




% ----------------- author(s) -------------------------

\begin{center}
{\bf Hans Petter Langtangen${}^{1, 2}$} \\ [0mm]
\end{center}

\begin{center}
% List of all institutions:
\centerline{{\small ${}^1$Simula Research Laboratory}}
\centerline{{\small ${}^2$University of Oslo}}
\end{center}
% ----------------- end author(s) -------------------------



% ----------------- date -------------------------


\begin{center}
Dec 31, 2012
\end{center}

\vspace{1cm}



\begin{itemize}
 \item When writing a note, report, manual, etc., do you find it difficult
   to choose the typesetting format? That is, to choose between plain
   (email-like) text, wiki, Word/OpenOffice, {\LaTeX}, HTML,
   reStructuredText, Sphinx, XML, etc.  Would it be convenient to
   start with some very simple text-like format that easily converts
   to the formats listed above, and then at some later stage
   eventually go with a particular format?

 \item Do you need to write documents in varying formats but find it
   difficult to remember all the typesetting details of various
   formats like \href{{http://refcards.com/docs/silvermanj/amslatex/LaTeXRefCard.v2.0.pdf}}{LaTeX}, \href{{http://www.htmlcodetutorial.com/}}{HTML}, \href{{http://docutils.sourceforge.net/docs/ref/rst/restructuredtext.html}}{reStructuredText}, \href{{http://sphinx.pocoo.org/contents.html}}{Sphinx}, and \href{{http://code.google.com/p/support/wiki/WikiSyntax}}{wiki}? Would it be convenient
   to generate the typesetting details of a particular format from a
   very simple text-like format with minimal tagging?

 \item Do you have the same information scattered around in different
   documents in different typesetting formats? Would it be a good idea
   to write things once, in one format, stored in one place, and
   include it anywhere?
\end{itemize}

\noindent
If any of these questions are of interest, you should keep on reading.


\section{What Does Doconce Look Like?}

Doconce text looks like ordinary text, but there are some almost invisible
text constructions that allow you to control the formating. Here are
som examples.

\begin{itemize}
  \item Bullet lists arise from lines starting with {\fontsize{10pt}{10pt}\Verb!*!}.

  \item \emph{Emphasized words} are surrounded by {\fontsize{10pt}{10pt}\Verb!*!}.

  \item \textbf{Words in boldface} are surrounded by underscores.

  \item Words from computer code are enclosed in back quotes and
    then typeset {\fontsize{10pt}{10pt}\Verb!verbatim (in a monospace font)!}.

  \item Section headings are recognied by equality ({\fontsize{10pt}{10pt}\Verb!=!}) signs before
    and after the title, and the number of {\fontsize{10pt}{10pt}\Verb!=!} signs indicates the
    level of the section: 7 for main section, 5 for subsection, and
    3 for subsubsection.

  \item Paragraph headings are recognized by a double underscore
    before and after the heading.

  \item The abstract of a document starts with \emph{Abstract} as paragraph
    heading, and all text up to the next heading makes up the abstract,

  \item Blocks of computer code can easily be included by placing
    {\fontsize{10pt}{10pt}\Verb!!bc!} (begin code) and {\fontsize{10pt}{10pt}\Verb!!ec!} (end code) commands at separate lines
    before and after the code block.

  \item Blocks of computer code can also be imported from source files.

  \item Blocks of {\LaTeX} mathematics can easily be included by placing
    {\fontsize{10pt}{10pt}\Verb!!bt!} (begin TeX) and {\fontsize{10pt}{10pt}\Verb!!et!} (end TeX) commands at separate lines
    before and after the math block.

  \item There is support for both {\LaTeX} and text-like inline mathematics.

  \item Figures and movies with captions, simple tables,
    URLs with links, index list, labels and references are supported.

  \item Invisible comments in the output format can be inserted throughout
    the text.

  \item Visible comments can be inserted so that authors and readers can
    comment upon the text (and at any time turn on/off output of such
    comments).

  \item There is an exercise environment with many advanced features.

  \item With a preprocessor, Preprocess or Mako, one can include
    other documents (files) and large portions of text can be defined
    in or out of the text.

  \item With Mako one can also have Python code
    embedded in the Doconce document and thereby parameterize the
    text (e.g., one text can describe programming in two languages).
\end{itemize}

\noindent
Here is an example of some simple text written in the Doconce format:
\begin{Verbatim}[fontsize=\fontsize{9pt}{9pt},tabsize=8,baselinestretch=0.85,
fontfamily=tt,xleftmargin=7mm]
===== A Subsection with Sample Text =====
label{my:first:sec}

Ordinary text looks like ordinary text, and the tags used for
_boldface_ words, *emphasized* words, and `computer` words look
natural in plain text.  Lists are typeset as you would do in email,

  * item 1
  * item 2
  * item 3

Lists can also have automatically numbered items instead of bullets,

  o item 1
  o item 2
  o item 3

URLs with a link word are possible, as in "hpl": "http://folk.uio.no/hpl".
If the word is URL, the URL itself becomes the link name,
as in "URL": "tutorial.do.txt".

References to sections may use logical names as labels (e.g., a
"label" command right after the section title), as in the reference to
Section ref{my:first:sec}.

Doconce also allows inline comments of the form [name: comment] (with
a space after `name:`), e.g., such as [hpl: here I will make some
remarks to the text]. Inline comments can be removed from the output
by a command-line argument (see Section ref{doconce2formats} for an
example).

Tables are also supperted, e.g.,

  |--------------------------------|
  |time  | velocity | acceleration |
  |---r-------r-----------r--------|
  | 0.0  | 1.4186   | -5.01        |
  | 2.0  | 1.376512 | 11.919       |
  | 4.0  | 1.1E+1   | 14.717624    |
  |--------------------------------|

# lines beginning with # are comment lines
\end{Verbatim}
\noindent
The Doconce text above results in the following little document:

\subsection{A Subsection with Sample Text}

\label{my:first:sec}

Ordinary text looks like ordinary text, and the tags used for
\textbf{boldface} words, \emph{emphasized} words, and {\fontsize{10pt}{10pt}\Verb!computer!} words look
natural in plain text.  Lists are typeset as you would do in an email,

\begin{itemize}
  \item item 1

  \item item 2

  \item item 3
\end{itemize}

\noindent
Lists can also have numbered items instead of bullets, just use an {\fontsize{10pt}{10pt}\Verb!o!}
(for ordered) instead of the asterisk:

\begin{enumerate}
 \item item 1

 \item item 2

 \item item 3
\end{enumerate}

\noindent
URLs with a link word are possible, as in \href{{http://folk.uio.no/hpl}}{hpl}.
If the word is URL, the URL itself becomes the link name,
as in \href{{tutorial.do.txt}}{\nolinkurl{tutorial.do.txt}}.

References to sections may use logical names as labels (e.g., a
"label" command right after the section title), as in the reference to
Section~\ref{my:first:sec}.

Doconce also allows inline comments such as \inlinecomment{hpl}{here I will make
some remarks to the text} for allowing authors to make notes. Inline
comments can be removed from the output by a command-line argument
(see Section~\ref{doconce2formats} for an example).

Tables are also supperted, e.g.,


\begin{quote}\begin{tabular}{rrr}
\hline
\multicolumn{1}{c}{ time } & \multicolumn{1}{c}{ velocity } & \multicolumn{1}{c}{ acceleration } \\
\hline
0.0          & 1.4186       & -5.01        \\
2.0          & 1.376512     & 11.919       \\
4.0          & 1.1E+1       & 14.717624    \\
\hline
\end{tabular}\end{quote}

\noindent

\subsection{Mathematics and Computer Code}

Inline mathematics, such as $\nu = \sin(x)$,
allows the formula to be specified both as {\LaTeX} and as plain text.
This results in a professional {\LaTeX} typesetting, but in other formats
the text version normally looks better than raw {\LaTeX} mathematics with
backslashes. An inline formula like $\nu = \sin(x)$ is
typeset as

\begin{Verbatim}[fontsize=\fontsize{9pt}{9pt},tabsize=8,baselinestretch=0.85,
fontfamily=tt,xleftmargin=7mm]
$\nu = \sin(x)$|$v = sin(x)$
\end{Verbatim}
\noindent
The pipe symbol acts as a delimiter between {\LaTeX} code and the plain text
version of the formula. If you write a lot of mathematics, only the
output formats {\fontsize{10pt}{10pt}\Verb!latex!}, {\fontsize{10pt}{10pt}\Verb!pdflatex!}, {\fontsize{10pt}{10pt}\Verb!html!}, {\fontsize{10pt}{10pt}\Verb!sphinx!}, and {\fontsize{10pt}{10pt}\Verb!pandoc!}
are of interest
and all these support inline {\LaTeX} mathematics so then you will naturally
drop the pipe symbol and write just

\begin{Verbatim}[fontsize=\fontsize{9pt}{9pt},tabsize=8,baselinestretch=0.85,
fontfamily=tt,xleftmargin=7mm]
$\nu = \sin(x)$
\end{Verbatim}
\noindent
However, if you want more textual formats, like plain text or reStructuredText,
the text after the pipe symbol may help to make the math formula more readable
if there are backslahes or other special {\LaTeX} symbols in the {\LaTeX} code.

Blocks of mathematics are typeset with raw {\LaTeX}, inside
{\fontsize{10pt}{10pt}\Verb!!bt!} and {\fontsize{10pt}{10pt}\Verb!!et!} (begin TeX, end TeX) instructions:

\begin{Verbatim}[fontsize=\fontsize{9pt}{9pt},tabsize=8,baselinestretch=0.85,
fontfamily=tt,xleftmargin=7mm]
!bt
\begin{align}
{\partial u\over\partial t} &= \nabla^2 u + f, label{myeq1}\\
{\partial v\over\partial t} &= \nabla\cdot(q(u)\nabla v) + g
\end{align}
!et
\end{Verbatim}
\noindent
% Note: !bt and !et (and !bc and !ec below) are used to illustrate
% tex and code blocks in inside verbatim blocks and are replaced
% by !bt, !et, !bc, and !ec after all other formatting is finished.
The result looks like this:

\begin{align}
{\partial u\over\partial t} &= \nabla^2 u + f, \label{myeq1}\\
{\partial v\over\partial t} &= \nabla\cdot(q(u)\nabla v) + g
\end{align}
Of course, such blocks only looks nice in formats with support
for {\LaTeX} mathematics, and here the align environment in particular
(this includes {\fontsize{10pt}{10pt}\Verb!latex!}, {\fontsize{10pt}{10pt}\Verb!pdflatex!}, {\fontsize{10pt}{10pt}\Verb!html!}, and {\fontsize{10pt}{10pt}\Verb!sphinx!}). The raw
{\LaTeX} syntax appears in simpler formats, but can still be useful
for those who can read {\LaTeX} syntax.

You can have blocks of computer code, starting and ending with
{\fontsize{10pt}{10pt}\Verb!!bc!} and {\fontsize{10pt}{10pt}\Verb!!ec!} instructions, respectively.

\begin{Verbatim}[fontsize=\fontsize{9pt}{9pt},tabsize=8,baselinestretch=0.85,
fontfamily=tt,xleftmargin=7mm]
!bc pycod
from math import sin, pi
def myfunc(x):
    return sin(pi*x)

import integrate
I = integrate.trapezoidal(myfunc, 0, pi, 100)
!ec
\end{Verbatim}
\noindent
Such blocks are formatted as

\begin{minted}[fontsize=\fontsize{9pt}{9pt},linenos=false,mathescape,baselinestretch=1.0,fontfamily=tt,xleftmargin=7mm]{python}
from math import sin, pi
def myfunc(x):
    return sin(pi*x)

import integrate
I = integrate.trapezoidal(myfunc, 0, pi, 100)
\end{minted}
\noindent
A code block must come after some plain sentence (at least for successful
output to {\fontsize{10pt}{10pt}\Verb!sphinx!}, {\fontsize{10pt}{10pt}\Verb!rst!}, and ASCII-close formats),
not directly after a section/paragraph heading or a table.


One can also copy computer code directly from files, either the
complete file or specified parts.  Computer code is then never
duplicated in the documentation (important for the principle of
avoiding copying information!).

Another document can be included by writing {\fontsize{10pt}{10pt}\Verb!# #include "mynote.do.txt"!}
at the beginning of a line.  Doconce documents have
extension {\fontsize{10pt}{10pt}\Verb!do.txt!}. The {\fontsize{10pt}{10pt}\Verb!do!} part stands for doconce, while the
trailing {\fontsize{10pt}{10pt}\Verb!.txt!} denotes a text document so that editors gives you
plain text editing capabilities.

\subsection{Macros (Newcommands), Cross-References, Index, and Bibliography}

\label{newcommands}

Doconce supports a type of macros via a LaTeX-style \emph{newcommand}
construction.  The newcommands defined in a file with name
{\fontsize{10pt}{10pt}\Verb!newcommand_replace.tex!} are expanded when Doconce is filtered to
other formats, except for {\LaTeX} (since {\LaTeX} performs the expansion
itself).  Newcommands in files with names {\fontsize{10pt}{10pt}\Verb!newcommands.tex!} and
{\fontsize{10pt}{10pt}\Verb!newcommands_keep.tex!} are kept unaltered when Doconce text is
filtered to other formats, except for the Sphinx format. Since Sphinx
understands {\LaTeX} math, but not newcommands if the Sphinx output is
HTML, it makes most sense to expand all newcommands.  Normally, a user
will put all newcommands that appear in math blocks surrounded by
{\fontsize{10pt}{10pt}\Verb!!bt!} and {\fontsize{10pt}{10pt}\Verb!!et!} in {\fontsize{10pt}{10pt}\Verb!newcommands_keep.tex!} to keep them unchanged, at
least if they contribute to make the raw {\LaTeX} math text easier to
read in the formats that cannot render {\LaTeX}.  Newcommands used
elsewhere throughout the text will usually be placed in
{\fontsize{10pt}{10pt}\Verb!newcommands_replace.tex!} and expanded by Doconce.  The definitions of
newcommands in the {\fontsize{10pt}{10pt}\Verb!newcommands*.tex!} files \emph{must} appear on a single
line (multi-line newcommands are too hard to parse with regular
expressions).

Recent versions of Doconce also offer cross referencing, typically one
can define labels below (sub)sections, in figure captions, or in
equations, and then refer to these later. Entries in an index can be
defined and result in an index at the end for the {\LaTeX} and Sphinx
formats. Citations to literature, with an accompanying bibliography in
a file, are also supported. The syntax of labels, references,
citations, and the bibliography closely resembles that of {\LaTeX},
making it easy for Doconce documents to be integrated in {\LaTeX}
projects (manuals, books). For further details on functionality and
syntax we refer to the {\fontsize{10pt}{10pt}\Verb!doc/manual/manual.do.txt!} file (see the
\href{{https://doconce.googlecode.com/hg/doc/demos/manual/index.html}}{demo page}
for various formats of this document).


% Example on including another Doconce file (using preprocess):


\section{From Doconce to Other Formats}

\label{doconce2formats}

Transformation of a Doconce document {\fontsize{10pt}{10pt}\Verb!mydoc.do.txt!} to various other
formats applies the script {\fontsize{10pt}{10pt}\Verb!doconce format!}:
\vspace{4pt}
\begin{Verbatim}[numbers=none,frame=lines,label=\fbox{{\tiny Terminal}},fontsize=\fontsize{9pt}{9pt},
labelposition=topline,framesep=2.5mm,framerule=0.7pt]
Terminal> doconce format format mydoc.do.txt
\end{Verbatim}
or just
\vspace{4pt}
\begin{Verbatim}[numbers=none,frame=lines,label=\fbox{{\tiny Terminal}},fontsize=\fontsize{9pt}{9pt},
labelposition=topline,framesep=2.5mm,framerule=0.7pt]
Terminal> doconce format format mydoc
\end{Verbatim}

\subsection{Preprocessing}

The {\fontsize{10pt}{10pt}\Verb!preprocess!} and {\fontsize{10pt}{10pt}\Verb!mako!} programs are used to preprocess the
file, and options to {\fontsize{10pt}{10pt}\Verb!preprocess!} and/or {\fontsize{10pt}{10pt}\Verb!mako!} can be added after the
filename. For example,
\vspace{4pt}
\begin{Verbatim}[numbers=none,frame=lines,label=\fbox{{\tiny Terminal}},fontsize=\fontsize{9pt}{9pt},
labelposition=topline,framesep=2.5mm,framerule=0.7pt]
Terminal> doconce format latex mydoc -Dextra_sections -DVAR1=5     # preprocess
Terminal> doconce format latex yourdoc extra_sections=True VAR1=5  # mako
\end{Verbatim}
The variable {\fontsize{10pt}{10pt}\Verb!FORMAT!} is always defined as the current format when
running {\fontsize{10pt}{10pt}\Verb!preprocess!} or {\fontsize{10pt}{10pt}\Verb!mako!}. That is, in the last example, {\fontsize{10pt}{10pt}\Verb!FORMAT!} is
defined as {\fontsize{10pt}{10pt}\Verb!latex!}. Inside the Doconce document one can then perform
format specific actions through tests like {\fontsize{10pt}{10pt}\Verb!#if FORMAT == "latex"!}
(for {\fontsize{10pt}{10pt}\Verb!preprocess!}) or {\fontsize{10pt}{10pt}\Verb!% if FORMAT == "latex":!} (for {\fontsize{10pt}{10pt}\Verb!mako!}).

\subsection{Removal of inline comments}

% mention notes also

The command-line arguments {\fontsize{10pt}{10pt}\Verb!--no-preprocess!} and {\fontsize{10pt}{10pt}\Verb!--no-mako!} turn off
running {\fontsize{10pt}{10pt}\Verb!preprocess!} and {\fontsize{10pt}{10pt}\Verb!mako!}, respectively.

Inline comments in the text are removed from the output by
\vspace{4pt}
\begin{Verbatim}[numbers=none,frame=lines,label=\fbox{{\tiny Terminal}},fontsize=\fontsize{9pt}{9pt},
labelposition=topline,framesep=2.5mm,framerule=0.7pt]
Terminal> doconce format latex mydoc --skip_inline_comments
\end{Verbatim}
One can also remove all such comments from the original Doconce
file by running:
\begin{Verbatim}[fontsize=\fontsize{9pt}{9pt},tabsize=8,baselinestretch=0.85,
fontfamily=tt,xleftmargin=7mm]
Terminal> doconce remove_inline_comments mydoc
\end{Verbatim}
\noindent
This action is convenient when a Doconce document reaches its final form
and comments by different authors should be removed.

\subsection{HTML}

Making an HTML version of a Doconce file {\fontsize{10pt}{10pt}\Verb!mydoc.do.txt!}
is performed by
\vspace{4pt}
\begin{Verbatim}[numbers=none,frame=lines,label=\fbox{{\tiny Terminal}},fontsize=\fontsize{9pt}{9pt},
labelposition=topline,framesep=2.5mm,framerule=0.7pt]
Terminal> doconce format html mydoc
\end{Verbatim}
The resulting file {\fontsize{10pt}{10pt}\Verb!mydoc.html!} can be loaded into any web browser for viewing.

The HTML style is defined in the header of the file. The default style
has blue section headings and white background. With the {\fontsize{10pt}{10pt}\Verb!--html-solarized!}
command line argument, the \href{{http://ethanschoonover.com/solarized}}{solarized}
color palette is used.

If the Pygments package (including the {\fontsize{10pt}{10pt}\Verb!pygmentize!} program)
is installed, code blocks are typeset with
aid of this package. The command-line argument {\fontsize{10pt}{10pt}\Verb!--no-pygments-html!}
turns off the use of Pygments and makes code blocks appear with
plain ({\fontsize{10pt}{10pt}\Verb!pre!}) HTML tags. The option {\fontsize{10pt}{10pt}\Verb!--pygments-html-linenos!} turns
on line numbers in Pygments-formatted code blocks.

The HTML file can be embedded in a template if the Doconce document
does not have a title (because then there will be
no header and footer in the HTML file). The template file must contain
valid HTML code and can have three "slots": {\fontsize{10pt}{10pt}\Verb!%(title)s!} for a title,
{\fontsize{10pt}{10pt}\Verb!%(date)s!} for a date, and {\fontsize{10pt}{10pt}\Verb!%(main)s!} for the main body of text, i.e., the
Doconce document translated to HTML. The title becomes the first
heading in the Doconce document, and the date is extracted from the
{\fontsize{10pt}{10pt}\Verb!DATE:!} line, if present. With the template feature one can easily embed
the text in the look and feel of a website. The template can be extracted
from the source code of a page at the site; just insert {\fontsize{10pt}{10pt}\Verb!%(title)s!} and
{\fontsize{10pt}{10pt}\Verb!%(date)s!} at appropriate places and replace the main bod of text
by {\fontsize{10pt}{10pt}\Verb!%(main)s!}. Here is an example:
\vspace{4pt}
\begin{Verbatim}[numbers=none,frame=lines,label=\fbox{{\tiny Terminal}},fontsize=\fontsize{9pt}{9pt},
labelposition=topline,framesep=2.5mm,framerule=0.7pt]
Terminal> doconce format html mydoc --html-template=mytemplate.html
\end{Verbatim}

\subsection{Pandoc and Markdown}

Output in Pandoc's extended Markdown format results from
\vspace{4pt}
\begin{Verbatim}[numbers=none,frame=lines,label=\fbox{{\tiny Terminal}},fontsize=\fontsize{9pt}{9pt},
labelposition=topline,framesep=2.5mm,framerule=0.7pt]
Terminal> doconce format pandoc mydoc
\end{Verbatim}
The name of the output file is {\fontsize{10pt}{10pt}\Verb!mydoc.mkd!}.
From this format one can go to numerous other formats:
\vspace{4pt}
\begin{Verbatim}[numbers=none,frame=lines,label=\fbox{{\tiny Terminal}},fontsize=\fontsize{9pt}{9pt},
labelposition=topline,framesep=2.5mm,framerule=0.7pt]
Terminal> pandoc -R -t mediawiki -o mydoc.mwk --toc mydoc.mkd
\end{Verbatim}
Pandoc supports {\fontsize{10pt}{10pt}\Verb!latex!}, {\fontsize{10pt}{10pt}\Verb!html!}, {\fontsize{10pt}{10pt}\Verb!odt!} (OpenOffice), {\fontsize{10pt}{10pt}\Verb!docx!} (Microsoft
Word), {\fontsize{10pt}{10pt}\Verb!rtf!}, {\fontsize{10pt}{10pt}\Verb!texinfo!}, to mention some. The {\fontsize{10pt}{10pt}\Verb!-R!} option makes
Pandoc pass raw HTML or {\LaTeX} to the output format instead of ignoring it,
while the {\fontsize{10pt}{10pt}\Verb!--toc!} option generates a table of contents.
See the \href{{http://johnmacfarlane.net/pandoc/README.html}}{Pandoc documentation}
for the many features of the {\fontsize{10pt}{10pt}\Verb!pandoc!} program.

Pandoc is useful to go from {\LaTeX} mathematics to, e.g., HTML or MS Word.
There are two ways (experiment to find the best one for your document):
{\fontsize{10pt}{10pt}\Verb!doconce format pandoc!} and then translating using {\fontsize{10pt}{10pt}\Verb!pandoc!}, or
{\fontsize{10pt}{10pt}\Verb!doconce format latex!}, and then going from {\LaTeX} to the desired format
using {\fontsize{10pt}{10pt}\Verb!pandoc!}.
Here is an example on the latter strategy:
\vspace{4pt}
\begin{Verbatim}[numbers=none,frame=lines,label=\fbox{{\tiny Terminal}},fontsize=\fontsize{9pt}{9pt},
labelposition=topline,framesep=2.5mm,framerule=0.7pt]
Terminal> doconce format latex mydoc
Terminal> doconce ptex2tex mydoc
Terminal> pandoc -f latex -t docx -o mydoc.docx mydoc.tex
\end{Verbatim}
When we go through {\fontsize{10pt}{10pt}\Verb!pandoc!}, only single equations or {\fontsize{10pt}{10pt}\Verb!align*!}
environments are well understood.

Quite some {\fontsize{10pt}{10pt}\Verb!doconce replace!} and {\fontsize{10pt}{10pt}\Verb!doconce subst!} edits might be needed
on the {\fontsize{10pt}{10pt}\Verb!.mkd!} or {\fontsize{10pt}{10pt}\Verb!.tex!} files to successfully have mathematics that is
well translated to MS Word.  Also when going to reStructuredText using
Pandoc, it can be advantageous to go via {\LaTeX}.

Here is an example where we take a Doconce snippet (without title, author,
and date), maybe with some unnumbered equations, and quickly generate
HTML with mathematics displayed my MathJax:
\vspace{4pt}
\begin{Verbatim}[numbers=none,frame=lines,label=\fbox{{\tiny Terminal}},fontsize=\fontsize{9pt}{9pt},
labelposition=topline,framesep=2.5mm,framerule=0.7pt]
Terminal> doconce format pandoc mydoc
Terminal> pandoc -t html -o mydoc.html -s --mathjax mydoc.mkd
\end{Verbatim}
The {\fontsize{10pt}{10pt}\Verb!-s!} option adds a proper header and footer to the {\fontsize{10pt}{10pt}\Verb!mydoc.html!} file.
This recipe is a quick way of makeing HTML notes with (some) mathematics.

\subsection{{\LaTeX}}

Making a {\LaTeX} file {\fontsize{10pt}{10pt}\Verb!mydoc.tex!} from {\fontsize{10pt}{10pt}\Verb!mydoc.do.txt!} is done in two steps:
% Note: putting code blocks inside a list is not successful in many
% formats - the text may be messed up. A better choice is a paragraph
% environment, as used here.

\paragraph{Step 1.}
Filter the doconce text to a pre-LaTeX form {\fontsize{10pt}{10pt}\Verb!mydoc.p.tex!} for
the {\fontsize{10pt}{10pt}\Verb!ptex2tex!} program (or {\fontsize{10pt}{10pt}\Verb!doconce ptex2tex!}):
\vspace{4pt}
\begin{Verbatim}[numbers=none,frame=lines,label=\fbox{{\tiny Terminal}},fontsize=\fontsize{9pt}{9pt},
labelposition=topline,framesep=2.5mm,framerule=0.7pt]
Terminal> doconce format latex mydoc
\end{Verbatim}
LaTeX-specific commands ("newcommands") in math formulas and similar
can be placed in files {\fontsize{10pt}{10pt}\Verb!newcommands.tex!}, {\fontsize{10pt}{10pt}\Verb!newcommands_keep.tex!}, or
{\fontsize{10pt}{10pt}\Verb!newcommands_replace.tex!} (see Section~\ref{newcommands}).
If these files are present, they are included in the {\LaTeX} document
so that your commands are defined.

An option {\fontsize{10pt}{10pt}\Verb!--latex-printed!} makes some adjustments for documents
aimed at being printed. For example, links to web resources are
associated with a footnote listing the complete web address (URL).

\paragraph{Step 2.}
Run {\fontsize{10pt}{10pt}\Verb!ptex2tex!} (if you have it) to make a standard {\LaTeX} file,
\vspace{4pt}
\begin{Verbatim}[numbers=none,frame=lines,label=\fbox{{\tiny Terminal}},fontsize=\fontsize{9pt}{9pt},
labelposition=topline,framesep=2.5mm,framerule=0.7pt]
Terminal> ptex2tex mydoc
\end{Verbatim}
In case you do not have {\fontsize{10pt}{10pt}\Verb!ptex2tex!}, you may run a (very) simplified version:
\vspace{4pt}
\begin{Verbatim}[numbers=none,frame=lines,label=\fbox{{\tiny Terminal}},fontsize=\fontsize{9pt}{9pt},
labelposition=topline,framesep=2.5mm,framerule=0.7pt]
Terminal> doconce ptex2tex mydoc
\end{Verbatim}

Note that Doconce generates a {\fontsize{10pt}{10pt}\Verb!.p.tex!} file with some preprocessor macros
that can be used to steer certain properties of the {\LaTeX} document.
For example, to turn on the Helvetica font instead of the standard
Computer Modern font, run
\vspace{4pt}
\begin{Verbatim}[numbers=none,frame=lines,label=\fbox{{\tiny Terminal}},fontsize=\fontsize{9pt}{9pt},
labelposition=topline,framesep=2.5mm,framerule=0.7pt]
Terminal> ptex2tex -DHELVETICA mydoc
Terminal> doconce ptex2tex mydoc -DHELVETICA  # alternative
\end{Verbatim}
The title, authors, and date are by default typeset in a non-standard
way to enable a nicer treatment of multiple authors having
institutions in common. However, the standard {\LaTeX} "maketitle" heading
is also available through {\fontsize{10pt}{10pt}\Verb!-DLATEX_HEADING=traditional!}.
A separate titlepage can be generate by
{\fontsize{10pt}{10pt}\Verb!-DLATEX_HEADING=titlepage!}.

Preprocessor variables to be defined or undefined are

\begin{itemize}
 \item {\fontsize{10pt}{10pt}\Verb!BOOK!} for the "book" documentclass rather than the standard
   "article" class (necessary if you apply chapter headings)

 \item {\fontsize{10pt}{10pt}\Verb!PALATINO!} for the Palatino font

 \item {\fontsize{10pt}{10pt}\Verb!HELVETIA!} for the Helvetica font

 \item {\fontsize{10pt}{10pt}\Verb!A4PAPER!} for A4 paper size

 \item {\fontsize{10pt}{10pt}\Verb!A6PAPER!} for A6 paper size (suitable for reading on small devices)

 \item {\fontsize{10pt}{10pt}\Verb!MOVIE15!} for using the movie15 {\LaTeX} package to display movies

 \item {\fontsize{10pt}{10pt}\Verb!PREAMBLE!} to turn the {\LaTeX} preamble on or off (i.e., complete document
   or document to be included elsewhere)

 \item {\fontsize{10pt}{10pt}\Verb!MINTED!} for inclusion of the minted package (which requires {\fontsize{10pt}{10pt}\Verb!latex!}
   or {\fontsize{10pt}{10pt}\Verb!pdflatex!} to be run with the {\fontsize{10pt}{10pt}\Verb!-shell-escape!} option)
\end{itemize}

\noindent
The {\fontsize{10pt}{10pt}\Verb!ptex2tex!} tool makes it possible to easily switch between many
different fancy formattings of computer or verbatim code in {\LaTeX}
documents. After any {\fontsize{10pt}{10pt}\Verb!!bc!} command in the Doconce source you can
insert verbatim block styles as defined in your {\fontsize{10pt}{10pt}\Verb!.ptex2tex.cfg!}
file, e.g., {\fontsize{10pt}{10pt}\Verb!!bc sys!} for a terminal session, where {\fontsize{10pt}{10pt}\Verb!sys!} is set to
a certain environment in {\fontsize{10pt}{10pt}\Verb!.ptex2tex.cfg!} (e.g., {\fontsize{10pt}{10pt}\Verb!CodeTerminal!}).
There are about 40 styles to choose from, and you can easily add
new ones.

Also the {\fontsize{10pt}{10pt}\Verb!doconce ptex2tex!} command supports preprocessor directives
for processing the {\fontsize{10pt}{10pt}\Verb!.p.tex!} file. The command allows specifications
of code environments as well. Here is an example:
\vspace{4pt}
\begin{Verbatim}[numbers=none,frame=lines,label=\fbox{{\tiny Terminal}},fontsize=\fontsize{9pt}{9pt},
labelposition=topline,framesep=2.5mm,framerule=0.7pt]
Terminal> doconce ptex2tex mydoc -DLATEX_HEADING=traditional \
          -DPALATINO -DA6PAPER \
          "sys=\begin{quote}\begin{verbatim}@\end{verbatim}\end{quote}" \
          fpro=minted fcod=minted shcod=Verbatim envir=ans:nt
\end{Verbatim}
Note that {\fontsize{10pt}{10pt}\Verb!@!} must be used to separate the begin and end {\LaTeX}
commands, unless only the environment name is given (such as {\fontsize{10pt}{10pt}\Verb!minted!}
above, which implies {\fontsize{10pt}{10pt}\Verb!\begin{minted}{fortran}!} and {\fontsize{10pt}{10pt}\Verb!\end{minted}!} as
begin and end for blocks inside {\fontsize{10pt}{10pt}\Verb!!bc fpro!} and {\fontsize{10pt}{10pt}\Verb!!ec!}).  Specifying
{\fontsize{10pt}{10pt}\Verb!envir=ans:nt!} means that all other environments are typeset with the
{\fontsize{10pt}{10pt}\Verb!anslistings.sty!} package, e.g., {\fontsize{10pt}{10pt}\Verb!!bc cppcod!} will then result in
{\fontsize{10pt}{10pt}\Verb!\begin{c++}!}. If no environments like {\fontsize{10pt}{10pt}\Verb!sys!}, {\fontsize{10pt}{10pt}\Verb!fpro!}, or the common
{\fontsize{10pt}{10pt}\Verb!envir!} are defined on the command line, the plain {\fontsize{10pt}{10pt}\Verb!\begin{verbatim}!}
and {\fontsize{10pt}{10pt}\Verb!\end{verbatim}!} used.


\paragraph{Step 2b (optional).}
Edit the {\fontsize{10pt}{10pt}\Verb!mydoc.tex!} file to your needs.
For example, you may want to substitute {\fontsize{10pt}{10pt}\Verb!section!} by {\fontsize{10pt}{10pt}\Verb!section*!} to
avoid numbering of sections, you may want to insert linebreaks
(and perhaps space) in the title, etc. This can be automatically
edited with the aid of the {\fontsize{10pt}{10pt}\Verb!doconce replace!} and {\fontsize{10pt}{10pt}\Verb!doconce subst!}
commands. The former works with substituting text directly, while the
latter performs substitutions using regular expressions.
Here are two examples:
\vspace{4pt}
\begin{Verbatim}[numbers=none,frame=lines,label=\fbox{{\tiny Terminal}},fontsize=\fontsize{9pt}{9pt},
labelposition=topline,framesep=2.5mm,framerule=0.7pt]
Terminal> doconce replace 'section{' 'section*{' mydoc.tex
Terminal> doconce subst 'title\{(.+)Using (.+)\}' \
          'title{\g<1> \\\\ [1.5mm] Using \g<2>' mydoc.tex
\end{Verbatim}
A lot of tailored fixes to the {\LaTeX} document can be done by
an appropriate set of text replacements and regular expression
substitutions. You are anyway encourged to make a script for
generating PDF from the {\LaTeX} file.

\paragraph{Step 3.}
Compile {\fontsize{10pt}{10pt}\Verb!mydoc.tex!}
and create the PDF file:
\vspace{4pt}
\begin{Verbatim}[numbers=none,frame=lines,label=\fbox{{\tiny Terminal}},fontsize=\fontsize{9pt}{9pt},
labelposition=topline,framesep=2.5mm,framerule=0.7pt]
Terminal> latex mydoc
Terminal> latex mydoc
Terminal> makeindex mydoc   # if index
Terminal> bibitem mydoc     # if bibliography
Terminal> latex mydoc
Terminal> dvipdf mydoc
\end{Verbatim}

If one wishes to run {\fontsize{10pt}{10pt}\Verb!ptex2tex!} and use the minted {\LaTeX} package for
typesetting code blocks ({\fontsize{10pt}{10pt}\Verb!Minted_Python!}, {\fontsize{10pt}{10pt}\Verb!Minted_Cpp!}, etc., in
{\fontsize{10pt}{10pt}\Verb!ptex2tex!} specified through the {\fontsize{10pt}{10pt}\Verb!*pro!} and {\fontsize{10pt}{10pt}\Verb!*cod!} variables in
{\fontsize{10pt}{10pt}\Verb!.ptex2tex.cfg!} or {\fontsize{10pt}{10pt}\Verb!$HOME/.ptex2tex.cfg!}), the minted {\LaTeX} package is
needed.  This package is included by running {\fontsize{10pt}{10pt}\Verb!ptex2tex!} with the
{\fontsize{10pt}{10pt}\Verb!-DMINTED!} option:
\vspace{4pt}
\begin{Verbatim}[numbers=none,frame=lines,label=\fbox{{\tiny Terminal}},fontsize=\fontsize{9pt}{9pt},
labelposition=topline,framesep=2.5mm,framerule=0.7pt]
Terminal> ptex2tex -DMINTED mydoc
\end{Verbatim}
In this case, {\fontsize{10pt}{10pt}\Verb!latex!} must be run with the
{\fontsize{10pt}{10pt}\Verb!-shell-escape!} option:
\vspace{4pt}
\begin{Verbatim}[numbers=none,frame=lines,label=\fbox{{\tiny Terminal}},fontsize=\fontsize{9pt}{9pt},
labelposition=topline,framesep=2.5mm,framerule=0.7pt]
Terminal> latex -shell-escape mydoc
Terminal> latex -shell-escape mydoc
Terminal> makeindex mydoc   # if index
Terminal> bibitem mydoc     # if bibliography
Terminal> latex -shell-escape mydoc
Terminal> dvipdf mydoc
\end{Verbatim}
When running {\fontsize{10pt}{10pt}\Verb!doconce ptex2tex mydoc envir=minted!} (or other minted
specifications with {\fontsize{10pt}{10pt}\Verb!doconce ptex2tex!}), the minted package is automatically
included so there is no need for the {\fontsize{10pt}{10pt}\Verb!-DMINTED!} option.

\subsection{PDFLaTeX}

Running {\fontsize{10pt}{10pt}\Verb!pdflatex!} instead of {\fontsize{10pt}{10pt}\Verb!latex!} follows almost the same steps,
but the start is
\vspace{4pt}
\begin{Verbatim}[numbers=none,frame=lines,label=\fbox{{\tiny Terminal}},fontsize=\fontsize{9pt}{9pt},
labelposition=topline,framesep=2.5mm,framerule=0.7pt]
Terminal> doconce format latex mydoc
\end{Verbatim}
Then {\fontsize{10pt}{10pt}\Verb!ptex2tex!} is run as explained above, and finally
\vspace{4pt}
\begin{Verbatim}[numbers=none,frame=lines,label=\fbox{{\tiny Terminal}},fontsize=\fontsize{9pt}{9pt},
labelposition=topline,framesep=2.5mm,framerule=0.7pt]
Terminal> pdflatex -shell-escape mydoc
Terminal> makeindex mydoc   # if index
Terminal> bibitem mydoc     # if bibliography
Terminal> pdflatex -shell-escape mydoc
\end{Verbatim}

\subsection{Plain ASCII Text}

We can go from Doconce "back to" plain untagged text suitable for viewing
in terminal windows, inclusion in email text, or for insertion in
computer source code:
\vspace{4pt}
\begin{Verbatim}[numbers=none,frame=lines,label=\fbox{{\tiny Terminal}},fontsize=\fontsize{9pt}{9pt},
labelposition=topline,framesep=2.5mm,framerule=0.7pt]
Terminal> doconce format plain mydoc.do.txt  # results in mydoc.txt
\end{Verbatim}

\subsection{reStructuredText}

Going from Doconce to reStructuredText gives a lot of possibilities to
go to other formats. First we filter the Doconce text to a
reStructuredText file {\fontsize{10pt}{10pt}\Verb!mydoc.rst!}:
\vspace{4pt}
\begin{Verbatim}[numbers=none,frame=lines,label=\fbox{{\tiny Terminal}},fontsize=\fontsize{9pt}{9pt},
labelposition=topline,framesep=2.5mm,framerule=0.7pt]
Terminal> doconce format rst mydoc.do.txt
\end{Verbatim}
We may now produce various other formats:
\vspace{4pt}
\begin{Verbatim}[numbers=none,frame=lines,label=\fbox{{\tiny Terminal}},fontsize=\fontsize{9pt}{9pt},
labelposition=topline,framesep=2.5mm,framerule=0.7pt]
Terminal> rst2html.py  mydoc.rst > mydoc.html # html
Terminal> rst2latex.py mydoc.rst > mydoc.tex  # latex
Terminal> rst2xml.py   mydoc.rst > mydoc.xml  # XML
Terminal> rst2odt.py   mydoc.rst > mydoc.odt  # OpenOffice
\end{Verbatim}

The OpenOffice file {\fontsize{10pt}{10pt}\Verb!mydoc.odt!} can be loaded into OpenOffice and
saved in, among other things, the RTF format or the Microsoft Word format.
However, it is more convenient to use the program {\fontsize{10pt}{10pt}\Verb!unovonv!}
to convert between the many formats OpenOffice supports \emph{on the command line}.
Run
\vspace{4pt}
\begin{Verbatim}[numbers=none,frame=lines,label=\fbox{{\tiny Terminal}},fontsize=\fontsize{9pt}{9pt},
labelposition=topline,framesep=2.5mm,framerule=0.7pt]
Terminal> unoconv --show
\end{Verbatim}
to see all the formats that are supported.
For example, the following commands take
{\fontsize{10pt}{10pt}\Verb!mydoc.odt!} to Microsoft Office Open XML format,
classic MS Word format, and PDF:
\vspace{4pt}
\begin{Verbatim}[numbers=none,frame=lines,label=\fbox{{\tiny Terminal}},fontsize=\fontsize{9pt}{9pt},
labelposition=topline,framesep=2.5mm,framerule=0.7pt]
Terminal> unoconv -f ooxml mydoc.odt
Terminal> unoconv -f doc mydoc.odt
Terminal> unoconv -f pdf mydoc.odt
\end{Verbatim}

\paragraph{Remark about Mathematical Typesetting.}
At the time of this writing, there is no easy way to go from Doconce
and {\LaTeX} mathematics to reST and further to OpenOffice and the
"MS Word world". Mathematics is only fully supported by {\fontsize{10pt}{10pt}\Verb!latex!} as
output and to a wide extent also supported by the {\fontsize{10pt}{10pt}\Verb!sphinx!} output format.
Some links for going from {\LaTeX} to Word are listed below.

\begin{itemize}
 \item \href{{http://ubuntuforums.org/showthread.php?t=1033441}}{\nolinkurl{http://ubuntuforums.org/showthread.php?t=1033441}}

 \item \href{{http://tug.org/utilities/texconv/textopc.html}}{\nolinkurl{http://tug.org/utilities/texconv/textopc.html}}

 \item \href{{http://nileshbansal.blogspot.com/2007/12/latex-to-openofficeword.html}}{\nolinkurl{http://nileshbansal.blogspot.com/2007/12/latex-to-openofficeword.html}}
\end{itemize}

\noindent

\subsection{Sphinx}

Sphinx documents demand quite some steps in their creation. We have automated
most of the steps through the {\fontsize{10pt}{10pt}\Verb!doconce sphinx_dir!} command:
\vspace{4pt}
\begin{Verbatim}[numbers=none,frame=lines,label=\fbox{{\tiny Terminal}},fontsize=\fontsize{9pt}{9pt},
labelposition=topline,framesep=2.5mm,framerule=0.7pt]
Terminal> doconce sphinx_dir author="authors' names" \
          title="some title" version=1.0 dirname=sphinxdir \
          theme=mytheme file1 file2 file3 ...
\end{Verbatim}
The keywords {\fontsize{10pt}{10pt}\Verb!author!}, {\fontsize{10pt}{10pt}\Verb!title!}, and {\fontsize{10pt}{10pt}\Verb!version!} are used in the headings
of the Sphinx document. By default, {\fontsize{10pt}{10pt}\Verb!version!} is 1.0 and the script
will try to deduce authors and title from the doconce files {\fontsize{10pt}{10pt}\Verb!file1!},
{\fontsize{10pt}{10pt}\Verb!file2!}, etc. that together represent the whole document. Note that
none of the individual Doconce files {\fontsize{10pt}{10pt}\Verb!file1!}, {\fontsize{10pt}{10pt}\Verb!file2!}, etc. should
include the rest as their union makes up the whole document.
The default value of {\fontsize{10pt}{10pt}\Verb!dirname!} is {\fontsize{10pt}{10pt}\Verb!sphinx-rootdir!}. The {\fontsize{10pt}{10pt}\Verb!theme!}
keyword is used to set the theme for design of HTML output from
Sphinx (the default theme is {\fontsize{10pt}{10pt}\Verb!'default'!}).

With a single-file document in {\fontsize{10pt}{10pt}\Verb!mydoc.do.txt!} one often just runs
\vspace{4pt}
\begin{Verbatim}[numbers=none,frame=lines,label=\fbox{{\tiny Terminal}},fontsize=\fontsize{9pt}{9pt},
labelposition=topline,framesep=2.5mm,framerule=0.7pt]
Terminal> doconce sphinx_dir mydoc
\end{Verbatim}
and then an appropriate Sphinx directory {\fontsize{10pt}{10pt}\Verb!sphinx-rootdir!} is made with
relevant files.

The {\fontsize{10pt}{10pt}\Verb!doconce sphinx_dir!} command generates a script
{\fontsize{10pt}{10pt}\Verb!automake_sphinx.py!} for compiling the Sphinx document into an HTML
document.  One can either run {\fontsize{10pt}{10pt}\Verb!automake_sphinx.py!} or perform the
steps in the script manually, possibly with necessary modifications.
You should at least read the script prior to executing it to have
some idea of what is done.

The {\fontsize{10pt}{10pt}\Verb!doconce sphinx_dir!} script copies directories named {\fontsize{10pt}{10pt}\Verb!figs!} or
{\fontsize{10pt}{10pt}\Verb!figures!} over to the Sphinx directory so that figures are accessible
in the Sphinx compilation.  If figures or movies are located in other
directories, {\fontsize{10pt}{10pt}\Verb!automake_sphinx.py!} must be edited accordingly.  Files,
to which there are local links (not {\fontsize{10pt}{10pt}\Verb!http:!} or {\fontsize{10pt}{10pt}\Verb!file:!} URLs), must be
placed in the {\fontsize{10pt}{10pt}\Verb!_static!} subdirectory of the Sphinx directory. The
utility {\fontsize{10pt}{10pt}\Verb!doconce sphinxfix_localURLs!} is run to check for local links
in the Doconce file: for each such link, say {\fontsize{10pt}{10pt}\Verb!dir1/dir2/myfile.txt!} it
replaces the link by {\fontsize{10pt}{10pt}\Verb!_static/myfile.txt!} and copies
{\fontsize{10pt}{10pt}\Verb!dir1/dir2/myfile.txt!} to a local {\fontsize{10pt}{10pt}\Verb!_static!} directory (in the same
directory as the script is run).  However, we recommend instead that
the writer of the document places files in {\fontsize{10pt}{10pt}\Verb!_static!} or lets a script
do it automatically. The user must copy all {\fontsize{10pt}{10pt}\Verb!_static/*!} files to the
{\fontsize{10pt}{10pt}\Verb!_static!} subdirectory of the Sphinx directory.  It may be wise to
always put files, to which there are local links in the Doconce
document, in a {\fontsize{10pt}{10pt}\Verb!_static!} or {\fontsize{10pt}{10pt}\Verb!_static-name!} directory and use these
local links. Then links do not need to be modified when creating a
Sphinx version of the document.

Doconce comes with a collection of HTML themes for Sphinx documents.
These are packed out in the Sphinx directory, the {\fontsize{10pt}{10pt}\Verb!conf.py!}
configuration file for Sphinx is edited accordingly, and a script
{\fontsize{10pt}{10pt}\Verb!make-themes.sh!} can make HTML documents with one or more themes.
For example,
to realize the themes {\fontsize{10pt}{10pt}\Verb!fenics!} and {\fontsize{10pt}{10pt}\Verb!pyramid!}, one writes
\vspace{4pt}
\begin{Verbatim}[numbers=none,frame=lines,label=\fbox{{\tiny Terminal}},fontsize=\fontsize{9pt}{9pt},
labelposition=topline,framesep=2.5mm,framerule=0.7pt]
Terminal> ./make-themes.sh fenics pyramid
\end{Verbatim}
The resulting directories with HTML documents are {\fontsize{10pt}{10pt}\Verb!_build/html_fenics!}
and {\fontsize{10pt}{10pt}\Verb!_build/html_pyramid!}, respectively. Without arguments,
{\fontsize{10pt}{10pt}\Verb!make-themes.sh!} makes all available themes (!).

If it is not desirable to use the autogenerated scripts explained
above, here is the complete manual procedure of generating a
Sphinx document from a file {\fontsize{10pt}{10pt}\Verb!mydoc.do.txt!}.

\paragraph{Step 1.}
Translate Doconce into the Sphinx format:
\vspace{4pt}
\begin{Verbatim}[numbers=none,frame=lines,label=\fbox{{\tiny Terminal}},fontsize=\fontsize{9pt}{9pt},
labelposition=topline,framesep=2.5mm,framerule=0.7pt]
Terminal> doconce format sphinx mydoc
\end{Verbatim}

\paragraph{Step 2.}
Create a Sphinx root directory
either manually or by using the interactive {\fontsize{10pt}{10pt}\Verb!sphinx-quickstart!}
program. Here is a scripted version of the steps with the latter:
\vspace{4pt}
\begin{Verbatim}[numbers=none,frame=lines,label=\fbox{{\tiny Terminal}},fontsize=\fontsize{9pt}{9pt},
labelposition=topline,framesep=2.5mm,framerule=0.7pt]
mkdir sphinx-rootdir
sphinx-quickstart <<EOF
sphinx-rootdir
n
_
Name of My Sphinx Document
Author
version
version
.rst
index
n
y
n
n
n
n
y
n
n
y
y
y
EOF
\end{Verbatim}
The autogenerated {\fontsize{10pt}{10pt}\Verb!conf.py!} file
may need some edits if you want to specific layout (Sphinx themes)
of HTML pages. The {\fontsize{10pt}{10pt}\Verb!doconce sphinx_dir!} generator makes an extended {\fontsize{10pt}{10pt}\Verb!conv.py!}
file where, among other things, several useful Sphinx extensions
are included.


\paragraph{Step 3.}
Copy the {\fontsize{10pt}{10pt}\Verb!mydoc.rst!} file to the Sphinx root directory:
\vspace{4pt}
\begin{Verbatim}[numbers=none,frame=lines,label=\fbox{{\tiny Terminal}},fontsize=\fontsize{9pt}{9pt},
labelposition=topline,framesep=2.5mm,framerule=0.7pt]
Terminal> cp mydoc.rst sphinx-rootdir
\end{Verbatim}
If you have figures in your document, the relative paths to those will
be invalid when you work with {\fontsize{10pt}{10pt}\Verb!mydoc.rst!} in the {\fontsize{10pt}{10pt}\Verb!sphinx-rootdir!}
directory. Either edit {\fontsize{10pt}{10pt}\Verb!mydoc.rst!} so that figure file paths are correct,
or simply copy your figure directories to {\fontsize{10pt}{10pt}\Verb!sphinx-rootdir!}.
Links to local files in {\fontsize{10pt}{10pt}\Verb!mydoc.rst!} must be modified to links to
files in the {\fontsize{10pt}{10pt}\Verb!_static!} directory, see comment above.

\paragraph{Step 4.}
Edit the generated {\fontsize{10pt}{10pt}\Verb!index.rst!} file so that {\fontsize{10pt}{10pt}\Verb!mydoc.rst!}
is included, i.e., add {\fontsize{10pt}{10pt}\Verb!mydoc!} to the {\fontsize{10pt}{10pt}\Verb!toctree!} section so that it becomes
\begin{Verbatim}[fontsize=\fontsize{9pt}{9pt},tabsize=8,baselinestretch=0.85,
fontfamily=tt,xleftmargin=7mm]
.. toctree::
   :maxdepth: 2

   mydoc
\end{Verbatim}
\noindent
(The spaces before {\fontsize{10pt}{10pt}\Verb!mydoc!} are important!)

\paragraph{Step 5.}
Generate, for instance, an HTML version of the Sphinx source:
\vspace{4pt}
\begin{Verbatim}[numbers=none,frame=lines,label=\fbox{{\tiny Terminal}},fontsize=\fontsize{9pt}{9pt},
labelposition=topline,framesep=2.5mm,framerule=0.7pt]
make clean   # remove old versions
make html
\end{Verbatim}

Sphinx can generate a range of different formats:
standalone HTML, HTML in separate directories with {\fontsize{10pt}{10pt}\Verb!index.html!} files,
a large single HTML file, JSON files, various help files (the qthelp, HTML,
and Devhelp projects), epub, {\LaTeX}, PDF (via {\LaTeX}), pure text, man pages,
and Texinfo files.

\paragraph{Step 6.}
View the result:
\vspace{4pt}
\begin{Verbatim}[numbers=none,frame=lines,label=\fbox{{\tiny Terminal}},fontsize=\fontsize{9pt}{9pt},
labelposition=topline,framesep=2.5mm,framerule=0.7pt]
Terminal> firefox _build/html/index.html
\end{Verbatim}

Note that verbatim code blocks can be typeset in a variety of ways
depending the argument that follows {\fontsize{10pt}{10pt}\Verb!!bc!}: {\fontsize{10pt}{10pt}\Verb!cod!} gives Python
({\fontsize{10pt}{10pt}\Verb!code-block:: python!} in Sphinx syntax) and {\fontsize{10pt}{10pt}\Verb!cppcod!} gives C++, but
all such arguments can be customized both for Sphinx and {\LaTeX} output.

\subsection{Wiki Formats}

There are many different wiki formats, but Doconce only supports three:
\href{{http://code.google.com/p/support/wiki/WikiSyntax}}{Googlecode wiki}, MediaWiki, and Creole Wiki. These formats are called
{\fontsize{10pt}{10pt}\Verb!gwiki!}, {\fontsize{10pt}{10pt}\Verb!mwiki!}, and {\fontsize{10pt}{10pt}\Verb!cwiki!}, respectively.
Transformation from Doconce to these formats is done by
\vspace{4pt}
\begin{Verbatim}[numbers=none,frame=lines,label=\fbox{{\tiny Terminal}},fontsize=\fontsize{9pt}{9pt},
labelposition=topline,framesep=2.5mm,framerule=0.7pt]
Terminal> doconce format gwiki mydoc.do.txt
Terminal> doconce format mwiki mydoc.do.txt
Terminal> doconce format cwiki mydoc.do.txt
\end{Verbatim}

The Googlecode wiki document, {\fontsize{10pt}{10pt}\Verb!mydoc.gwiki!}, is most conveniently stored
in a directory which is a clone of the wiki part of the Googlecode project.
This is far easier than copying and pasting the entire text into the
wiki editor in a web browser.

When the Doconce file contains figures, each figure filename must in
the {\fontsize{10pt}{10pt}\Verb!.gwiki!} file be replaced by a URL where the figure is
available. There are instructions in the file for doing this. Usually,
one performs this substitution automatically (see next section).

From the MediaWiki format one can go to other formats with aid
of \href{{http://pediapress.com/code/}}{mwlib}. This means that one can
easily use Doconce to write \href{{http://en.wikibooks.org}}{Wikibooks}
and publish these in PDF and MediaWiki format.
At the same time, the book can also be published as a
standard {\LaTeX} book or a Sphinx web document.

\subsection{Tweaking the Doconce Output}

Occasionally, one would like to tweak the output in a certain format
from Doconce. One example is figure filenames when transforming
Doconce to reStructuredText. Since Doconce does not know if the
{\fontsize{10pt}{10pt}\Verb!.rst!} file is going to be filtered to {\LaTeX} or HTML, it cannot know
if {\fontsize{10pt}{10pt}\Verb!.eps!} or {\fontsize{10pt}{10pt}\Verb!.png!} is the most appropriate image filename.
The solution is to use a text substitution command or code with, e.g., sed,
perl, python, or scitools subst, to automatically edit the output file
from Doconce. It is then wise to run Doconce and the editing commands
from a script to automate all steps in going from Doconce to the final
format(s). The {\fontsize{10pt}{10pt}\Verb!make.sh!} files in {\fontsize{10pt}{10pt}\Verb!docs/manual!} and {\fontsize{10pt}{10pt}\Verb!docs/tutorial!}
constitute comprehensive examples on how such scripts can be made.

\subsection{Demos}

The current text is generated from a Doconce format stored in the file
\begin{Verbatim}[fontsize=\fontsize{9pt}{9pt},tabsize=8,baselinestretch=0.85,
fontfamily=tt,xleftmargin=7mm]
docs/tutorial/tutorial.do.txt
\end{Verbatim}
\noindent
The file {\fontsize{10pt}{10pt}\Verb!make.sh!} in the {\fontsize{10pt}{10pt}\Verb!tutorial!} directory of the
Doconce source code contains a demo of how to produce a variety of
formats.  The source of this tutorial, {\fontsize{10pt}{10pt}\Verb!tutorial.do.txt!} is the
starting point.  Running {\fontsize{10pt}{10pt}\Verb!make.sh!} and studying the various generated
files and comparing them with the original {\fontsize{10pt}{10pt}\Verb!tutorial.do.txt!} file,
gives a quick introduction to how Doconce is used in a real case.
\href{{https://doconce.googlecode.com/hg/doc/demos/tutorial/index.html}}{Here}
is a sample of how this tutorial looks in different formats.

There is another demo in the {\fontsize{10pt}{10pt}\Verb!docs/manual!} directory which
translates the more comprehensive documentation, {\fontsize{10pt}{10pt}\Verb!manual.do.txt!}, to
various formats. The {\fontsize{10pt}{10pt}\Verb!make.sh!} script runs a set of translations.


\section{Installation of Doconce and its Dependencies}

\subsection{Doconce}

Doconce itself is pure Python code hosted at \href{{http://code.google.com/p/doconce}}{\nolinkurl{http://code.google.com/p/doconce}}.  Its installation from the
Mercurial ({\fontsize{10pt}{10pt}\Verb!hg!}) source follows the standard procedure:
\vspace{4pt}
\begin{Verbatim}[numbers=none,frame=lines,label=\fbox{{\tiny Terminal}},fontsize=\fontsize{9pt}{9pt},
labelposition=topline,framesep=2.5mm,framerule=0.7pt]
# Doconce
hg clone https://doconce.googlecode.com/hg/ doconce
cd doconce
sudo python setup.py install
cd ..
\end{Verbatim}
Since Doconce is frequently updated, it is recommended to use the
above procedure and whenever a problem occurs, make sure to
update to the most recent version:
\vspace{4pt}
\begin{Verbatim}[numbers=none,frame=lines,label=\fbox{{\tiny Terminal}},fontsize=\fontsize{9pt}{9pt},
labelposition=topline,framesep=2.5mm,framerule=0.7pt]
cd doconce
hg pull
hg update
sudo python setup.py install
\end{Verbatim}

Debian GNU/Linux users can also run
\vspace{4pt}
\begin{Verbatim}[numbers=none,frame=lines,label=\fbox{{\tiny Terminal}},fontsize=\fontsize{9pt}{9pt},
labelposition=topline,framesep=2.5mm,framerule=0.7pt]
sudo apt-get install doconce
\end{Verbatim}
This installs the latest release and not the most updated and bugfixed
version.
On Ubuntu one needs to run
\vspace{4pt}
\begin{Verbatim}[numbers=none,frame=lines,label=\fbox{{\tiny Terminal}},fontsize=\fontsize{9pt}{9pt},
labelposition=topline,framesep=2.5mm,framerule=0.7pt]
sudo add-apt-repository ppa:scitools/ppa
sudo apt-get update
sudo apt-get install doconce
\end{Verbatim}

\subsection{Dependencies}

\paragraph{Preprocessors.}
If you make use of the \href{{http://code.google.com/p/preprocess}}{Preprocess}
preprocessor, this program must be installed:

\vspace{4pt}
\begin{Verbatim}[numbers=none,frame=lines,label=\fbox{{\tiny Terminal}},fontsize=\fontsize{9pt}{9pt},
labelposition=topline,framesep=2.5mm,framerule=0.7pt]
svn checkout http://preprocess.googlecode.com/svn/trunk/ preprocess
cd preprocess
cd doconce
sudo python setup.py install
cd ..
\end{Verbatim}

A much more advanced alternative to Preprocess is
\href{{http://www.makotemplates.org}}{Mako}. Its installation is most
conveniently done by {\fontsize{10pt}{10pt}\Verb!pip!},

\vspace{4pt}
\begin{Verbatim}[numbers=none,frame=lines,label=\fbox{{\tiny Terminal}},fontsize=\fontsize{9pt}{9pt},
labelposition=topline,framesep=2.5mm,framerule=0.7pt]
pip install Mako
\end{Verbatim}
This command requires {\fontsize{10pt}{10pt}\Verb!pip!} to be installed. On Debian Linux systems,
such as Ubuntu, the installation is simply done by

\vspace{4pt}
\begin{Verbatim}[numbers=none,frame=lines,label=\fbox{{\tiny Terminal}},fontsize=\fontsize{9pt}{9pt},
labelposition=topline,framesep=2.5mm,framerule=0.7pt]
sudo apt-get install python-pip
\end{Verbatim}
Alternatively, one can install from the {\fontsize{10pt}{10pt}\Verb!pip!} \href{{http://pypi.python.org/pypi/pip}}{source code}.

Mako can also be installed directly from
\href{{http://www.makotemplates.org/download.html}}{source}: download the
tarball, pack it out, go to the directory and run
the usual {\fontsize{10pt}{10pt}\Verb!sudo python setup.py install!}.

\paragraph{Image file handling.}
Different output formats require different formats of image files.
For example, PostScript or Encapuslated PostScript is required for {\fontsize{10pt}{10pt}\Verb!latex!}
output, while HTML needs JPEG, GIF, or PNG formats.
Doconce calls up programs from the ImageMagick suite for converting
image files to a proper format if needed. The \href{{http://www.imagemagick.org/script/index.php}}{ImageMagick suite} can be installed on all major platforms.
On Debian Linux (including Ubuntu) systems one can simply write

\vspace{4pt}
\begin{Verbatim}[numbers=none,frame=lines,label=\fbox{{\tiny Terminal}},fontsize=\fontsize{9pt}{9pt},
labelposition=topline,framesep=2.5mm,framerule=0.7pt]
sudo apt-get install imagemagick
\end{Verbatim}

The convenience program {\fontsize{10pt}{10pt}\Verb!doconce combine_images!}, for combining several
images into one, will use {\fontsize{10pt}{10pt}\Verb!montage!} and {\fontsize{10pt}{10pt}\Verb!convert!} from ImageMagick and
the {\fontsize{10pt}{10pt}\Verb!pdftk!}, {\fontsize{10pt}{10pt}\Verb!pdfnup!}, and {\fontsize{10pt}{10pt}\Verb!pdfcrop!} programs from the {\fontsize{10pt}{10pt}\Verb!texlive-extra-utils!}
Debian package. The latter gets installed by

\vspace{4pt}
\begin{Verbatim}[numbers=none,frame=lines,label=\fbox{{\tiny Terminal}},fontsize=\fontsize{9pt}{9pt},
labelposition=topline,framesep=2.5mm,framerule=0.7pt]
sudo apt-get install texlive-extra-utils
\end{Verbatim}

\paragraph{Spellcheck.}
The utility {\fontsize{10pt}{10pt}\Verb!doconce spellcheck!} applies the {\fontsize{10pt}{10pt}\Verb!ispell!} program for
spellcheck. On Debian (including Ubuntu) it is installed by

\vspace{4pt}
\begin{Verbatim}[numbers=none,frame=lines,label=\fbox{{\tiny Terminal}},fontsize=\fontsize{9pt}{9pt},
labelposition=topline,framesep=2.5mm,framerule=0.7pt]
sudo apt-get install ispell
\end{Verbatim}

\paragraph{Ptex2tex for {\LaTeX} Output.}
To make {\LaTeX} documents with very flexible choice of typesetting of
verbatim code blocks you need \href{{http://code.google.com/p/ptex2tex}}{ptex2tex},
which is installed by

\vspace{4pt}
\begin{Verbatim}[numbers=none,frame=lines,label=\fbox{{\tiny Terminal}},fontsize=\fontsize{9pt}{9pt},
labelposition=topline,framesep=2.5mm,framerule=0.7pt]
svn checkout http://ptex2tex.googlecode.com/svn/trunk/ ptex2tex
cd ptex2tex
sudo python setup.py install
\end{Verbatim}
It may happen that you need additional style files, you can run
a script, {\fontsize{10pt}{10pt}\Verb!cp2texmf.sh!}:

\vspace{4pt}
\begin{Verbatim}[numbers=none,frame=lines,label=\fbox{{\tiny Terminal}},fontsize=\fontsize{9pt}{9pt},
labelposition=topline,framesep=2.5mm,framerule=0.7pt]
cd latex
sh cp2texmf.sh  # copy stylefiles to ~/texmf directory
cd ../..
\end{Verbatim}
This script copies some special stylefiles that
that {\fontsize{10pt}{10pt}\Verb!ptex2tex!} potentially makes use of. Some more standard stylefiles
are also needed. These are installed by

\vspace{4pt}
\begin{Verbatim}[numbers=none,frame=lines,label=\fbox{{\tiny Terminal}},fontsize=\fontsize{9pt}{9pt},
labelposition=topline,framesep=2.5mm,framerule=0.7pt]
sudo apt-get install texlive-latex-recommended texlive-latex-extra
\end{Verbatim}
on Debian Linux (including Ubuntu) systems. TeXShop on Mac comes with
the necessary stylefiles (if not, they can be found by googling and installed
manually in the {\fontsize{10pt}{10pt}\Verb!~/texmf/tex/latex/misc!} directory).

Note that the {\fontsize{10pt}{10pt}\Verb!doconce ptex2tex!} command, which needs no installation
beyond Doconce itself, can be used as a simpler alternative to the {\fontsize{10pt}{10pt}\Verb!ptex2tex!}
program.

The \emph{minted} {\LaTeX} style is offered by {\fontsize{10pt}{10pt}\Verb!ptex2tex!} and {\fontsize{10pt}{10pt}\Verb!doconce ptext2tex!}
is popular among many
users. This style requires the package \href{{http://pygments.org}}{Pygments}
to be installed. On Debian Linux,
\vspace{4pt}
\begin{Verbatim}[numbers=none,frame=lines,label=\fbox{{\tiny Terminal}},fontsize=\fontsize{9pt}{9pt},
labelposition=topline,framesep=2.5mm,framerule=0.7pt]
sudo apt-get install python-pygments
\end{Verbatim}
Alternatively, the package can be installed manually:
\vspace{4pt}
\begin{Verbatim}[numbers=none,frame=lines,label=\fbox{{\tiny Terminal}},fontsize=\fontsize{9pt}{9pt},
labelposition=topline,framesep=2.5mm,framerule=0.7pt]
hg clone ssh://hg@bitbucket.org/birkenfeld/pygments-main pygments
cd pygments
sudo python setup.py install
\end{Verbatim}

If you use the minted style together with {\fontsize{10pt}{10pt}\Verb!ptex2tex!}, you have to
enable it by the {\fontsize{10pt}{10pt}\Verb!-DMINTED!} command-line argument to {\fontsize{10pt}{10pt}\Verb!ptex2tex!}.
This is not necessary if you run the alternative {\fontsize{10pt}{10pt}\Verb!doconce ptex2tex!} program.

All
use of the minted style requires the {\fontsize{10pt}{10pt}\Verb!-shell-escape!} command-line
argument when running {\LaTeX}, i.e., {\fontsize{10pt}{10pt}\Verb!latex -shell-escape!} or {\fontsize{10pt}{10pt}\Verb!pdflatex -shell-escape!}.

% Say something about anslistings.sty

\paragraph{reStructuredText (reST) Output.}
The {\fontsize{10pt}{10pt}\Verb!rst!} output from Doconce allows further transformation to {\LaTeX},
HTML, XML, OpenOffice, and so on, through the \href{{http://docutils.sourceforge.net}}{docutils} package.  The installation of the
most recent version can be done by

\vspace{4pt}
\begin{Verbatim}[numbers=none,frame=lines,label=\fbox{{\tiny Terminal}},fontsize=\fontsize{9pt}{9pt},
labelposition=topline,framesep=2.5mm,framerule=0.7pt]
svn checkout http://docutils.svn.sourceforge.net/svnroot/docutils/trunk/docutils
cd docutils
sudo python setup.py install
cd ..
\end{Verbatim}
To use the OpenOffice suite you will typically on Debian systems install
\vspace{4pt}
\begin{Verbatim}[numbers=none,frame=lines,label=\fbox{{\tiny Terminal}},fontsize=\fontsize{9pt}{9pt},
labelposition=topline,framesep=2.5mm,framerule=0.7pt]
sudo apt-get install unovonv libreoffice libreoffice-dmaths
\end{Verbatim}

There is a possibility to create PDF files from reST documents
using ReportLab instead of {\LaTeX}. The enabling software is
\href{{http://code.google.com/p/rst2pdf}}{rst2pdf}. Either download the tarball
or clone the svn repository, go to the {\fontsize{10pt}{10pt}\Verb!rst2pdf!} directory and
run the usual {\fontsize{10pt}{10pt}\Verb!sudo python setup.py install!}.


Output to {\fontsize{10pt}{10pt}\Verb!sphinx!} requires of course the
\href{{http://sphinx.pocoo.org}}{Sphinx software},
installed by

\vspace{4pt}
\begin{Verbatim}[numbers=none,frame=lines,label=\fbox{{\tiny Terminal}},fontsize=\fontsize{9pt}{9pt},
labelposition=topline,framesep=2.5mm,framerule=0.7pt]
hg clone https://bitbucket.org/birkenfeld/sphinx
cd sphinx
sudo python setup.py install
cd ..
\end{Verbatim}

\paragraph{Markdown and Pandoc Output.}
The Doconce format {\fontsize{10pt}{10pt}\Verb!pandoc!} outputs the document in the Pandoc
extended Markdown format, which via the {\fontsize{10pt}{10pt}\Verb!pandoc!} program can be
translated to a range of other formats. Installation of \href{{http://johnmacfarlane.net/pandoc/}}{Pandoc}, written in Haskell, is most
easily done by

\vspace{4pt}
\begin{Verbatim}[numbers=none,frame=lines,label=\fbox{{\tiny Terminal}},fontsize=\fontsize{9pt}{9pt},
labelposition=topline,framesep=2.5mm,framerule=0.7pt]
sudo apt-get install pandoc
\end{Verbatim}
on Debian (Ubuntu) systems.

\paragraph{Epydoc Output.}
When the output format is {\fontsize{10pt}{10pt}\Verb!epydoc!} one needs that program too, installed
by
\vspace{4pt}
\begin{Verbatim}[numbers=none,frame=lines,label=\fbox{{\tiny Terminal}},fontsize=\fontsize{9pt}{9pt},
labelposition=topline,framesep=2.5mm,framerule=0.7pt]
svn co https://epydoc.svn.sourceforge.net/svnroot/epydoc/trunk/epydoc epydoc
cd epydoc
sudo make install
cd ..
\end{Verbatim}

\paragraph{Remark.}
Several of the packages above installed from source code
are also available in Debian-based system through the
{\fontsize{10pt}{10pt}\Verb!apt-get install!} command. However, we recommend installation directly
from the version control system repository as there might be important
updates and bug fixes. For {\fontsize{10pt}{10pt}\Verb!svn!} directories, go to the directory,
run {\fontsize{10pt}{10pt}\Verb!svn update!}, and then {\fontsize{10pt}{10pt}\Verb!sudo python setup.py install!}. For
Mercurial ({\fontsize{10pt}{10pt}\Verb!hg!}) directories, go to the directory, run
{\fontsize{10pt}{10pt}\Verb!hg pull; hg update!}, and then {\fontsize{10pt}{10pt}\Verb!sudo python setup.py install!}.

\printindex

\end{document}